\documentclass{article}
\usepackage{amsmath, amsthm, amssymb, amsfonts}
\usepackage{thmtools}
\usepackage{graphicx}
\usepackage{setspace}
\usepackage{geometry}
\usepackage{float}
\usepackage{hyperref}
\usepackage[utf8]{inputenc}
\usepackage[english]{babel}
\usepackage{framed}
\usepackage[dvipsnames]{xcolor}
\usepackage{tcolorbox}

\colorlet{LightGray}{White!90!Periwinkle}
\colorlet{LightOrange}{Orange!15}
\colorlet{LightGreen}{Green!15}

\newcommand{\HRule}[1]{\rule{\linewidth}{#1}}

\colorlet{LightGray}{black!10}
\colorlet{LightOrange}{orange!15}
\colorlet{LightGreen}{green!15}
\colorlet{LightBlue}{blue!15}
\colorlet{LightCyan}{cyan!15}



\declaretheoremstyle[name=Theorem,]{thmsty}
\declaretheorem[style=thmsty,numberwithin=section]{theorem}
\usepackage{tcolorbox} % Add missing package
\tcolorboxenvironment{theorem}{colback=LightGray}

\declaretheoremstyle[name=Definition,]{thmsty}
\declaretheorem[style=thmsty,numberwithin=section]{definition}
\tcolorboxenvironment{definition}{colback=LightBlue}

\declaretheoremstyle[name=Proposition,]{prosty}
\declaretheorem[style=prosty,numberlike=theorem]{proposition}
\tcolorboxenvironment{proposition}{colback=LightOrange}


\declaretheoremstyle[name=Proof,]{prosty}
\declaretheorem[style=prosty,numberlike=theorem]{proofbox}
\tcolorboxenvironment{proofbox}{colback=LightOrange}

\declaretheoremstyle[name=Axiom,]{prcpsty}
\declaretheorem[style=prcpsty,numberlike=theorem]{axiom}
\tcolorboxenvironment{axiom}{colback=LightGreen}

\declaretheoremstyle[name=Lemma,]{prcpsty}
\declaretheorem[style=prcpsty,numberlike=theorem]{lemma}
\tcolorboxenvironment{lemma}{colback=LightCyan}





\setstretch{1.2}
\geometry{
    textheight=9in,
    textwidth=5.5in,
    top=1in,
    headheight=12pt,
    headsep=25pt,
    footskip=30pt
}

% ------------------------------------------------------------------------------

\begin{document}

% ------------------------------------------------------------------------------
% Cover Page and ToC
% ------------------------------------------------------------------------------

\title{ \normalsize \textsc{}
		\\ [2.0cm]
		\HRule{1.5pt} \\
		\LARGE \textbf{\uppercase{Lecture 1 }}
		\HRule{2.0pt} \\ [0.6cm] \LARGE{}
		}

\date{\today}
\author{\textbf{Author} \\ 
		Tom Jeong
        }

\maketitle
\newpage

\tableofcontents
\newpage

% ------------------------------------------------------------------------------

\section{Relations}
    Let $S$ be s a set (e.g. $\mathbb{N } = \{1, 2, 3, \dots\}$), integers: $\mathbb{Z} = \{\dots, -2, -1, 0, 1, 2, \dots\}$ 
    \\ 
    rational numbers $\mathbb{Q}$  i should know these .. 
 \\ 
 \underline{RECALL}: cartesian product: $A$ x $B = \{(a,b)  | a \in A, b \in B\}$  \\ 
 \begin{definition}
    A \underline{relation} R on a set $S$ is a subset $R \subseteq S \times S$. we write $xRy$ (read "x is related to y") to denote an element $(x,y) \in R$

 \end{definition} some examples of relations: "is orthogonal to" is a relation o the set of vectors in $\mathbb{R}^3$ \\ another example "is congruent to" is a relation on the set of polygons \\ 
    "divides" is a relation on the set of integers

WE say the relation R is 
\begin{enumerate}
    \item \underline{reflexive} if $xRx$ for all $x \in S$
    \item \underline{symmetric} if $xRy$ implies $yRx$ for all $x,y \in S$
    \item \underline{antisymmetric} if $xRy$ and $yRx$ implies $x=y$ for all $x,y \in S$
    \item \underline{transitive} if $xRy$ and $yRz$ implies $xRz$ for all $x,y,z \in S$
\end{enumerate} (sometimes we note $\sim$ for the relation R) \\
 going back to examples: "is congruent to" is reflexive, symmetric, and transitive \\ 

 \section{equivalence and partial order relations}
 \begin{definition}
    two very important types of relations are: 
    \begin{enumerate}
        \item \underline{equivalence relations} 
            \begin{enumerate}
                \item reflextive
                \item symmetric
                \item transitive
            \end{enumerate}
        \item \underline{partial order relations}
            \begin{enumerate}
                \item reflexive
                \item \underline{antisymmetric}
                \item transitive
            \end{enumerate}
    \end{enumerate}
\end{definition}
    e.g. $R = \{(a,b) | b-a \in \mathbb{N}\} \subseteq \mathbb{Z} \times \mathbb{Z}$ 
    \\
    is R an equivalence relation? a partial ordering ? 
    \begin{enumerate}
        \item reflexive: $aRa$? yes since $a - a = 0 \in \mathbb{N}$
        \item antisymmetric: $aRb$ and $bRa$ implies $a=b$? if $aRb$ then $b-a \in \mathbb{N}$ and if $bRa$ then $a-b \in \mathbb{N}$ so $a$ and $b$ are the same
        \item transitive: $aRb$ and $bRc$ implies $aRc$? if $aRb$ then $b-a \in \mathbb{N}$ and if $bRc$ then $c-b \in \mathbb{N}$ so $c-a = (c-b) + (b-a) \in \mathbb{N}$
    \end{enumerate}
    \underline{Exercise}: check that $aRb$ iff $a \leq b$ 


    e.g. fix $n \in \mathbb{Z}^{+}$  $R = \{(a,b) | a -b \in n \mathbb{Z} \} \subseteq \mathbb{Z} \times \mathbb{Z} $ \\ 
    \underline{Exercise} this is an equivalece relation 
    \\ $aRb$ if $a- b = nk$ for some $k \in \mathbb{Z}$ \\ \underline{we call this relation congruence module n}

     \underline{notation:}if $R$ is an equivalence relation, we often denote it by $\sim$ 

    \begin{definition}[equivalence class]
        If $\sim$ is an equivalene relation on $S$ we can fix and element $x \in S$ and consider all elements of $S$ that is equivalent to $x$: \\ 
        $$[x] = \{s \in S| s \sim x\} \subseteq S$$ 
        and x is a representative of this equivalence class
    \end{definition} 
    e.g. $S = \{a, b, c, d, e\}$ and $\sim = \{(a,a), (b,b), (c,c), (d,d), (e,e), (a,b), (b,a), (c,d), (d,c), (d,e), (e,d), (c,e), (e,c) \}$ \\
    $[a] = \{a,b\} = [b] $  and 
    $[c] = \{c,d,e\} = [d] = [e]$ \\  \\ 
    Q: are equivalence classes always disjoint? \\ 
    A: yes, if $[x] \cap [y] \neq \emptyset$ then $[x] = [y]$ \\
    \begin{lemma}
        Let $\sim$ be an equivalene relation on $S$ and $x,y \in S$. Then $[x] = [y]$ iff $x \sim y$
    \end{lemma}
    \begin{proof}[proof of lemma 2.1] 
        iff \\ 
        $\rightarrow$: supose $[x] = [y]$ then $x \in [x] $, since $\sim$ is reflexive and so $x \in [y]$ Thus, $x \sim y$
        \\ 
        $\leftarrow$: Let $s \in [x]$ since $x \sim x$ and $x \sim y$ (by assumption) then $s \sim y$ since it is transitive. hence $s \in [y]$ and so $[x] \subseteq [y]$
        \\ similarly, if $s \in [y]$, then $s \sim y$ and $x \sim y$ so since $\sim$ is symmetric and trasnstiive $s \sim x$ and so $x in [x]$ hence $[y] \subseteq [x]$
    \end{proof}

    \underline{Corollary}: $$[x] \cap [y] = \emptyset \text{ if } [x] \not = [y]$$
    \begin{proof}
        suppose $[x] \cap [y] \neq \emptyset$ then there exists $s \in [x] \cap [y]$ then $s \sim x$ and $s \sim y$ by lemma 2.1 $x \sim y$ and so $[x] = [y]$
    \end{proof}

    \begin{definition}
        a \underline{partition} on a set $S$ is a collection $(S_i)_{i \in I}$ of nonempty subsets of $S$ such that
        \begin{enumerate}
            \item $S = \bigcup_{i \in I} S_i$
            \item $S_i \cap S_j = \emptyset$ for all $i,j \in I$ with $i \not = j$
        \end{enumerate}
    \end{definition}
    e.g. $S = \{a,b,c,d,e\}$ and $S_1 = \{a,b\}$, $S_2 = \{c,d,e\}$ is a partition of $S$ \\

    \begin{theorem}
        Let $S$ be a set with an equivalece relation $\sim$: Then the set of equivalence classes $S / \sim = \{[x] | x \in S\}$ is a partition of $S$
        \\ 
        conversely, if $(S_i)_{i \in I}$ is a partition of $S$ then there exists an equivalence relation $\sim$ on $S$ such that $S / \sim = (S_i)_{i \in I}$
    \end{theorem}
    \begin{proof}
        $\rightarrow$ we will first show that the set of equivalence classes is a partition, we already showed $[x] \cap [y] = \emptyset$ if $[x] \not = [y]$ so we need to show that $S = \bigcup_{x \in S} [x]$ \\
        let $s \in S$ then $s \in [s]$ so $s \in \bigcup_{x \in S} [x]$ and so $S \subseteq \bigcup_{x \in S} [x]$ \\ 
        For each $x \in S, [x] \subseteq S$ so $\bigcup_{x \in S} [x] \subseteq S$ and so $S = \bigcup_{x \in S} [x]$ \\
        \\ $\leftarrow$ now suppose $(S_i)_{i \in I}$ is a partition of $S$. Define $x \sim y$ iff $x,y \in S_i$ for some $i$. We have $\sim$ is reflexive since each $x \in S_i$ for some $i \in I$. As $\bigcup_{i \in I} S_i = S$. 
        \\ $\sim$ is symmetric since containm ent in a set have an order 
        \\ $\sim$ is transitive since if $x,y \in S_i$ and $y,z \in S_j$ then $x,y,z \in S_i \cap S_j = \emptyset$ so $x \sim z$
    \end{proof}
    e.g. $S = \mathbb{Z}$, $\sim$ is $\equiv \mod n$
    \\ $S / \sim = \{[0], [1], \dots, [n-1]\}$ is a partition of $\mathbb{Z}$
    \begin{align}
        [0] &= \{\dots, -2n, -n, 0, n, 2n, \dots\} \\ 
        [1] &= \{\dots, -2n+1, -n+1, 1, n+1, 2n+1, \dots\} \\ 
        &\vdots \\ 
        [n-1] &= \{\dots, -2n + n-1, -n + n-1, n-1, 2n + n-1, \dots\}
    \end{align}

\section{Partial orders }
\underline{notation:} when a relation $R$ is a partial order, we often denote it by $\leq$ \\
e.g. $\leq$ is a partial order on $mathbbR$ \\
Let $S$ be a set and $\mathbb{P}(s)$ its power set. $\leq$ is a partial order on $\mathbb{P}(S)$ \\ 
ex: "Divides" is a partial order on $\mathbb{N}$ (but not on $\mathbb{Z}$)
\begin{definition}[ minimal and first element ]
    let $\leq$ be ap artial order on a set $S$. An element $s \in S$ is \underline{minimal} if $ x\leq s \rightarrow x = s \forall x \in S$. \\ 
    An element $t \in S$ is called a \underline{first} element if $ t \leq x \forall x \in S$ .. (Warning: not every minimal element is a first element)
\end{definition}
    e.g. $S = \{ 2, 3, 4, 5, 6 \dots \}$ divides is a partial order on $S$ 
    \\ 2 is a minimal element. 3 is a minimal element . 5 is a minimal element. every prime number is a minimal element  
    But they are not  first elements 
    \begin{proposition}
        \begin{enumerate}
            \item A first element is unique 
            \item a first element is a minimal element 
            \item a minimal element needs not to be a first element*** 
        \end{enumerate}
    \end{proposition}
\end{document}
