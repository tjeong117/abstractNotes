\documentclass[10pt,twoside]{article}

\usepackage{amssymb,amsmath,amsthm,amsfonts, epsfig, graphicx, dsfont,
  bbm, bbold, url, color, setspace, multirow,pinlabel}
\usepackage[all]{xy}
\usepackage{amsmath, amsthm, amssymb, amsfonts}
\usepackage{thmtools}
\usepackage{graphicx}
\usepackage{setspace}
\usepackage{geometry}
\usepackage{float}
\usepackage{hyperref}
\usepackage[utf8]{inputenc}
\usepackage[english]{babel}
\usepackage{framed}
\usepackage[dvipsnames]{xcolor}
\usepackage{tcolorbox}
\usepackage{tikz}

\usepackage{fancyhdr} \setlength{\voffset}{-1in}
\setlength{\topmargin}{0in} \setlength{\textheight}{9.5in}
\setlength{\textwidth}{6.5in} \setlength{\hoffset}{0in}
\setlength{\oddsidemargin}{0in} \setlength{\evensidemargin}{0in}
\setlength{\marginparsep}{0in} \setlength{\marginparwidth}{0in}
\setlength{\headsep}{0.25in} \setlength{\headheight}{0.5in}
\pagestyle{fancy}

\newcommand\R{\mathbb{R}}
\newcommand\C{\mathbb{C}}
\newcommand\Z{\mathbb{Z}}

\colorlet{LightGray}{black!10}
\colorlet{LightOrange}{orange!15}
\colorlet{LightGreen}{green!15}
\colorlet{LightBlue}{blue!15}
\colorlet{LightCyan}{cyan!15}

\declaretheoremstyle[name=Solution,]{thmsty}
\declaretheorem[style=thmsty,numberwithin=section]{solution}
\tcolorboxenvironment{solution}{colback=LightBlue}

\fancyhead[LO,LE]{Math 4107 - Hom} \fancyhead[RO,RE]{Due 9/6/24 at 5 pm}
\chead{\textbf{}} \cfoot{}
\fancyfoot[LO,LE]{} \fancyfoot[RO,RE]{Page \thepage\ of
  \pageref{LastPage}} \renewcommand{\footrulewidth}{0.5pt}
\parindent 0in
%% ------------------------------------------------------%%
%% -------------------Begin Document---------------------%%
%% ------------------------------------------------------%%

\begin{document}

\begin{center}
\huge{\bf{Homework 3}}
\end{center}

I encourage you to work on the assignment with other students; remember to name anyone you collaborate with and to write up your final solutions by yourself.\\

\begin{itemize}
	\item Lauritzen Chapter 2, Exercises 2, 4, 5, 9, 10
	\item Problems not from the textbook
  	\begin{enumerate}
		\item Prove or disprove that the following are groups:
		 \begin{enumerate}
		 	\item even integers under addition
		 	\begin{solution}
                Let $G$ be the set of even integers under addition. We need to check the group axioms: \\ 
                \textbf{Closure:} Let $a, b \in G$. Then $a = 2m$ and $b = 2n$ for some integers $m$ and $n$. Then $a + b = 2m + 2n = 2(m + n)$, which is even. Thus, $a + b \in G$. \\
                \textbf{Associativity:} Addition of integers is associative, so this holds. \\
                \textbf{Identity:} The identity element is 0, which is even. \\
                \textbf{Inverse:} Let $a \in G$. Then $a = 2m$ for some integer $m$. Then $-a = -2m = 2(-m)$, which is even. Thus, $-a \in G$. \\
                Thus, $G$ is a group.

            \end{solution}
			\item odd integers under addition
			\begin{solution}
                Let $G$ be the set of odd integers under addition. We need to check the group axioms: \\ 
                \textbf{Closure:} Let $a, b \in G$. Then $a = 2m + 1$ and $b = 2n + 1$ for some integers $m$ and $n$. Then $a + b = 2m + 1 + 2n + 1 = 2(m + n) + 2 = 2(m + n + 1)$, which is even. Thus, $a + b \notin G$. \\
                Thus, $G$ is not a group.
                
            \end{solution}
			\item $\{ 3^n : n \in \Z \}$ under multiplication.
			\begin{solution}
                Let $G = \{ 3^n : n \in \Z \}$ under multiplication. We need to check the group axioms: \\ 
                \textbf{Closure:} Let $a, b \in G$. Then $a = 3^m$ and $b = 3^n$ for some integers $m$ and $n$. Then $a \cdot b = 3^m \cdot 3^n = 3^{m + n}$, which is in $G$. Thus, $a \cdot b \in G$. \\
                \textbf{Associativity:} Multiplication of integers is associative, so this holds. \\
                \textbf{Identity:} The identity element is 1, which is in $G$. \\
                \textbf{Inverse:} Let $a \in G$. Then $a = 3^m$ for some integer $m$. Then $a^{-1} = 3^{-m} = \frac{1}{3^m}$, which is in $G$. \\
                Thus, $G$ is a group.

            \end{solution}
		 \end{enumerate}		
 		 \item Matrix groups are an extremely important class of groups. Assume all matrices in the following statements have entries in $\R$, though it is fun to think about the same questions for matrices with entries in $\Z$ or $\C$! It is also fun to think about the same questions for $n \times n$ matrices. You may assume that matrix addition and matrix multiplication are associative. Prove or disprove that the following are groups:
		 \begin{enumerate}
		 	\item $2 \times 2$ diagonal matrices under matrix addition
		 	\begin{solution}
                Let $G$ be the set of $2 \times 2$ diagonal matrices under matrix addition. We need to check the group axioms: \\ 
                \textbf{Closure:} Let $A, B \in G$. Then $A = \begin{pmatrix} a & 0 \\ 0 & b \end{pmatrix}$ and $B = \begin{pmatrix} c & 0 \\ 0 & d \end{pmatrix}$ for some real numbers $a, b, c, d$. Then $A + B = \begin{pmatrix} a + c & 0 \\ 0 & b + d \end{pmatrix}$, which is in $G$. Thus, $A + B \in G$. \\
                \textbf{Associativity:} Matrix addition is associative, so this holds. \\
                \textbf{Identity:} The identity element is the $2 \times 2$ zero matrix, which is in $G$. \\
                \textbf{Inverse:} Let $A \in G$. Then $A = \begin{pmatrix} a & 0 \\ 0 & b \end{pmatrix}$ for some real numbers $a, b$. Then $-A = \begin{pmatrix} -a & 0 \\ 0 & -b \end{pmatrix}$, which is in $G$. \\
                Thus, $G$ is a group.
            \end{solution}
			\item $2 \times 2$ matrices with determinant 1 under matrix addition
			\begin{solution}
                Let $G$ be the set of $2 \times 2$ matrices with determinant 1 under matrix addition. We need to check the group axioms: \\ 
                \textbf{Closure:} Let $A, B \in G$. Then $\det(A) = 1$ and $\det(B) = 1$. Then $\det(A + B) = \det(A) + \det(B) = 1 + 1 = 2 \neq 1$. Thus, $A + B \notin G$. \\
                Thus, $G$ is not a group.

            \end{solution}
			\item $2 \times 2$ matrices with determinant 1 under matrix multiplication
			\begin{solution}
                Let $G$ be the set of $2 \times 2$ matrices with determinant 1 under matrix multiplication. We need to check the group axioms: \\ 
                \textbf{Closure:} Let $A, B \in G$. Then $\det(A) = 1$ and $\det(B) = 1$. Then $\det(A \cdot B) = \det(A) \cdot \det(B) = 1 \cdot 1 = 1$. Thus, $A \cdot B \in G$. \\
                \textbf{Associativity:} Matrix multiplication is associative, so this holds. \\
                \textbf{Identity:} The identity element is the $2 \times 2$ identity matrix, which is in $G$. \\
                \textbf{Inverse:} Let $A \in G$. Then $\det(A) = 1$. Then $A^{-1} = \frac{1}{\det(A)} \cdot \text{adj}(A)$, which is in $G$. \\
                Thus, $G$ is a group.
            \end{solution}
			\item $2 \times 2$ matrices with trace 0 under matrix addition
			\begin{solution}
                Let $G$ be the set of $2 \times 2$ matrices with trace 0 under matrix addition. We need to check the group axioms: \\ 
                \textbf{Closure:} Let $A, B \in G$. Then $\text{tr}(A) = 0$ and $\text{tr}(B) = 0$. Then $\text{tr}(A + B) = \text{tr}(A) + \text{tr}(B) = 0 + 0 = 0$. Thus, $A + B \in G$. \\
                \textbf{Associativity:} Matrix addition is associative, so this holds. \\
                \textbf{Identity:} The identity element is the $2 \times 2$ zero matrix, which is in $G$. \\
                \textbf{Inverse:} Let $A \in G$. Then $\text{tr}(A) = 0$. Then $-A = \begin{pmatrix} -a & -b \\ -c & -d \end{pmatrix}$, which is in $G$. \\
                Thus, $G$ is a group.
            \end{solution}
			\item $2 \times 2$ symmetric matrices (recall that a matrix $A$ is symmetric if $A = A^T$) with nonzero determinant under matrix multiplication
			\begin{solution}
                Let $G$ be the set of $2 \times 2$ symmetric matrices with nonzero determinant under matrix multiplication. We need to check the group axioms: \\ 
                \textbf{Closure:} Let $A, B \in G$. Then $A = \begin{pmatrix} a & b \\ b & c \end{pmatrix}$ and $B = \begin{pmatrix} d & e \\ e & f \end{pmatrix}$ for some real numbers $a, b, c, d, e, f$. Then $A \cdot B = \begin{pmatrix} ad + be & ae + bf \\ bd + ce & be + cf \end{pmatrix}$, which is in $G$. Thus, $A \cdot B \in G$. \\
                \textbf{Associativity:} Matrix multiplication is associative, so this holds. \\
                \textbf{Identity:} The identity element is the $2 \times 2$ identity matrix, which is in $G$. \\
                \textbf{Inverse:} Let $A \in G$. Then $\det(A) \neq 0$. Then $A^{-1} = \frac{1}{\det(A)} \cdot \text{adj}(A)$, which is in $G$. \\
                Thus, $G$ is a group.
            \end{solution}
			\item $2 \times 2$ orthogonal matrices (recall that a matrix $A$ is orthogonal if $A A^T = I$, where $I$ is the identity matrix) under matrix multplication.
			\begin{solution}
                Let $G$ be the set of $2 \times 2$ orthogonal matrices under matrix multiplication. We need to check the group axioms: \\
                \textbf{Closure:} Let $A, B \in G$. Then $A A^T = I$ and $B B^T = I$. Then $(A \cdot B) (A \cdot B)^T = A B B^T A^T = A I A^T = A A^T = I$. Thus, $A \cdot B \in G$. \\
                \textbf{Associativity:} Matrix multiplication is associative, so this holds. \\
                \textbf{Identity:} The identity element is the $2 \times 2$ identity matrix, which is in $G$. \\
                \textbf{Inverse:} Let $A \in G$. Then $A A^T = I$. Then $A^{-1} = A^T$, which is in $G$. \\
                Thus, $G$ is a group.
            \end{solution}
		 \end{enumerate}
		\item Recall Euler's formula $e^{i \theta} = \cos \theta + i \sin \theta$. Let $n$ be a positive integer. The complex equation
		\[ z^n = 1\]
		can then be written as $e^{ni \theta} = \cos (n \theta) + \sin (n \theta) = 1$. Notice this has solutions of the form $\theta = \frac{2k \pi}{n}$ where $k \in \Z$. Thus, we see the equation $z^n =1$ has $n$ distinct solutions, 
		\[ z_k = e^{\frac{2k i \pi}{n}} = \cos \Big(\frac{2k \pi}{n}\Big) + i \sin \Big(\frac{2k \pi}{n}\Big), \]
where $0 \leq k < n$. These are called the $n^{\textup{th}}$ roots of unity. Prove that the set of $n^{\textup{th}}$ roots of unity under the operation of complex multiplication forms a group. You may assume that complex multiplication is associated. Sketch the $n^{\textup{th}}$ roots of unity in the complex plane for $n=2, 3, 4$.
	
   
       \textbf{Proof:}
Let \( G_n = \left\{ z_k = e^{\frac{2k\pi i}{n}} : 0 \leq k < n \right\} \) be the set of \( n^{\text{th}} \) roots of unity. We need to prove that \( (G_n, \cdot) \) is a group, where \( \cdot \) denotes complex multiplication.

\begin{enumerate}
\item \textbf{Closure:} For any \( z_j, z_k \in G_n \), we need to show that \( z_j \cdot z_k \in G_n \).
\begin{align*}
z_j \cdot z_k &= e^{\frac{2j\pi i}{n}} \cdot e^{\frac{2k\pi i}{n}} \\
&= e^{\frac{2(j+k)\pi i}{n}} \\
&= e^{\frac{2m\pi i}{n}}, \text{ where } m = (j+k) \mod n
\end{align*}
Since \( 0 \leq m < n \), \( z_j \cdot z_k \in G_n \).

\item \textbf{Associativity:} This is given in the problem statement.

\item \textbf{Identity Element:} The identity element is \( e^0 = 1 \), which is in \( G_n \) (it's \( z_0 \)). For any \( z_k \in G_n \):
\[
z_k \cdot 1 = 1 \cdot z_k = z_k.
\]

\item \textbf{Inverse Element:} For any \( z_k \in G_n \), its inverse is \( z_{n-k} \) (or \( z_0 \) if \( k=0 \)).
\begin{align*}
z_k \cdot z_{n-k} &= e^{\frac{2k\pi i}{n}} \cdot e^{\frac{2(n-k)\pi i}{n}} \\
&= e^{\frac{2n\pi i}{n}} \\
&= e^{2\pi i} = 1.
\end{align*}

\end{enumerate}

Therefore, \( (G_n, \cdot) \) is a group.

\textbf{Sketches of \( n^{\text{th}} \) Roots of Unity:}

For \( n=2 \) (Square roots of unity):
\begin{tikzpicture}
\draw[->] (-1.5,0) -- (1.5,0) node[right] {Re};
\draw[->] (0,-1.5) -- (0,1.5) node[above] {Im};
\fill (1,0) circle (2pt) node[above right] {1};
\fill (-1,0) circle (2pt) node[above left] {-1};
\end{tikzpicture}

For \( n=3 \) (Cube roots of unity):
\begin{tikzpicture}
\draw[->] (-1.5,0) -- (1.5,0) node[right] {Re};
\draw[->] (0,-1.5) -- (0,1.5) node[above] {Im};
\fill (1,0) circle (2pt) node[right] {1};
\fill (-0.5,0.866) circle (2pt) node[above left] {$e^{2\pi i/3}$};
\fill (-0.5,-0.866) circle (2pt) node[below left] {$e^{4\pi i/3}$};
\end{tikzpicture}

For \( n=4 \) (Fourth roots of unity):
\begin{tikzpicture}
\draw[->] (-1.5,0) -- (1.5,0) node[right] {Re};
\draw[->] (0,-1.5) -- (0,1.5) node[above] {Im};
\fill (1,0) circle (2pt) node[right] {1};
\fill (0,1) circle (2pt) node[above] {$i$};
\fill (-1,0) circle (2pt) node[left] {-1};
\fill (0,-1) circle (2pt) node[below] {$-i$};
\end{tikzpicture}
   

\end{enumerate}
\end{itemize}


\label{LastPage}
\end{document}
