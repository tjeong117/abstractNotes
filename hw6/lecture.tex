\documentclass{article}
\usepackage{amsmath, amsthm, amssymb, amsfonts}
\usepackage{thmtools}
\usepackage{graphicx}
\usepackage{setspace}
\usepackage{geometry}
\usepackage{float}
\usepackage{hyperref}
\usepackage[utf8]{inputenc}
\usepackage[english]{babel}
\usepackage{framed}
\usepackage[dvipsnames]{xcolor}
\usepackage{tcolorbox}

\colorlet{LightGray}{White!90!Periwinkle}
\colorlet{LightOrange}{Orange!15}
\colorlet{LightGreen}{Green!15}

\newcommand{\HRule}[1]{\rule{\linewidth}{#1}}

\colorlet{LightGray}{black!10}
\colorlet{LightOrange}{orange!15}
\colorlet{LightGreen}{green!15}
\colorlet{LightBlue}{blue!15}
\colorlet{LightCyan}{cyan!15}



\declaretheoremstyle[name=Theorem,]{thmsty}
\declaretheorem[style=thmsty,numberwithin=section]{theorem}
\usepackage{tcolorbox} % Add missing package
\tcolorboxenvironment{theorem}{colback=LightGray}

\declaretheoremstyle[name=Definition,]{thmsty}
\declaretheorem[style=thmsty,numberwithin=section]{definition}
\tcolorboxenvironment{definition}{colback=LightBlue}

\declaretheoremstyle[name=Proposition,]{prosty}
\declaretheorem[style=prosty,numberlike=theorem]{proposition}
\tcolorboxenvironment{proposition}{colback=LightOrange}


\declaretheoremstyle[name=Proof,]{prosty}
\declaretheorem[style=prosty,numberlike=theorem]{proofbox}
\tcolorboxenvironment{proofbox}{colback=LightOrange}

\declaretheoremstyle[name=Axiom,]{prcpsty}
\declaretheorem[style=prcpsty,numberlike=theorem]{axiom}
\tcolorboxenvironment{axiom}{colback=LightGreen}

\declaretheoremstyle[name=Lemma,]{prcpsty}
\declaretheorem[style=prcpsty,numberlike=theorem]{lemma}
\tcolorboxenvironment{lemma}{colback=LightCyan}





\setstretch{1.2}
\geometry{
    textheight=9in,
    textwidth=5.5in,
    top=1in,
    headheight=12pt,
    headsep=25pt,
    footskip=30pt
}

% ------------------------------------------------------------------------------

\begin{document}

% ------------------------------------------------------------------------------
% Cover Page and ToC
% ------------------------------------------------------------------------------
% Exercises 30, 33, 34, 35, 36


\title{hw6}
\author{Tom Jeong}
% ------------------------------------------------------------------------------
\section*{Exercise 30} 
Let $\pi: G \to G/N $ be a canonical group homomorphism. whree N is a normal subgroup of G  \begin{enumerate}
    \item prove that $\pi(K)$ is a subgroup of $G/N$ if  $K$ is a subgroup of $G$ 
    \item prove that $\pi^{-1}(H)$ is a subgroup of $G$ containing N if H is a subgroup of $G/N$
    \item prove that $\pi(\pi^{-1}(H)) = H$ and $\pi^{-1}(\pi(K)) = K$ where H is a subgroup of $G/N$ and K is a subgroup of G containing N 
    \item Let G be a cyclic group and $f: G \to K $ a surjective group homomorphism. Prove that K is cyclic.
    \item let $n \in \mathbb{n}$ prove using the canonical group homomorphism $\pi: \mathbb{Z} \to \mathbb{Z} / N\mathbb{Z}$ that subgroup of H of $\mathbb{Z} / N\mathbb{Z}$ is cyclic 
\end{enumerate}
\begin{proof}
    \begin{enumerate}
        \item Let $K$ be a subgroup of $G$. We need to show that $\pi(K)$ is a subgroup of $G/N$. Since $K$ is a subgroup of $G$, it is non-empty. Let $x, y \in \pi(K)$. Then, there exist $a, b \in K$ such that $\pi(a) = x$ and $\pi(b) = y$. Since $K$ is a subgroup of $G$, $ab \in K$. Therefore, $\pi(ab) = \pi(a)\pi(b) = xy \in \pi(K)$. Also, since $K$ is a subgroup of $G$, $a^{-1} \in K$. Therefore, $\pi(a^{-1}) = \pi(a)^{-1} = x^{-1} \in \pi(K)$. Hence, $\pi(K)$ is a subgroup of $G/N$.
        \item Let $H$ be a subgroup of $G/N$. We need to show that $\pi^{-1}(H)$ is a subgroup of $G$ containing $N$. Since $H$ is a subgroup of $G/N$, it is non-empty. Let $x, y \in \pi^{-1}(H)$. Then, there exist $a, b \in G$ such that $\pi(a) = x$ and $\pi(b) = y$. Since $H$ is a subgroup of $G/N$, $xy \in H$. Therefore, $\pi(ab) = \pi(a)\pi(b) = xy \in H$. Also, since $H$ is a subgroup of $G/N$, $x^{-1} \in H$. Therefore, $\pi(a^{-1}) = \pi(a)^{-1} = x^{-1} \in H$. Hence, $\pi^{-1}(H)$ is a subgroup of $G$ containing $N$.
        \item Let $H$ be a subgroup of $G/N$ and $K$ be a subgroup of $G$ containing $N$. We need to show that $\pi(\pi^{-1}(H)) = H$ and $\pi^{-1}(\pi(K)) = K$. Let $x \in \pi(\pi^{-1}(H))$. Then, there exists $a \in G$ such that $\pi(a) = x$. Since $a \in \pi^{-1}(H)$, $\pi(a) \in H$. Therefore, $x \in H$. Hence, $\pi(\pi^{-1}(H)) = H$. Let $y \in \pi^{-1}(\pi(K))$. Then, there exists $b \in G$ such that $\pi(b) = y$. Since $b \in \pi(K)$, $\pi(b) \in \pi(K)$. Therefore, $y \in K$. Hence, $\pi^{-1}(\pi(K)) = K$.
        \item Let $G$ be a cyclic group and $f: G \to K$ be a surjective group homomorphism. We need to show that $K$ is cyclic. Since $G$ is cyclic, there exists $a \in G$ such that $G = \langle a \rangle$. Since $f$ is surjective, $K = f(G) = f(\langle a \rangle) = \langle f(a) \rangle$. Hence, $K$ is cyclic.
        \item Let $n \in \mathbb{N}$. We need to prove using the canonical group homomorphism $\pi: \mathbb{Z} \to \mathbb{Z}/N\mathbb{Z}$ that every subgroup $H$ of $\mathbb{Z}/N\mathbb{Z}$ is cyclic. Since $\mathbb{Z}$ is cyclic, there exists $a \in \mathbb{Z}$ such that $\mathbb{Z} = \langle a \rangle$. Since $\pi$ is surjective, $\mathbb{Z}/N\mathbb{Z} = \pi(\mathbb{Z}) = \pi(\langle a \rangle) = \langle \pi(a) \rangle$. Hence, $\mathbb{Z}/N\mathbb{Z}$ is cyclic.
        
    \end{enumerate}
\end{proof}

% ------------------------------------------------------------------------------
\section*{Exercise 33}
Consider \( Z \subset \mathbb{Q} \) as abelian groups with \( + \) as composition. Let \( [q] = q + Z \in \mathbb{Q}/Z \), where \( q \in \mathbb{Q} \).

\begin{enumerate}
    \item show that $[\frac{9}{4}]$ has order 4 in $\mathbb{Q}/Z$
    \begin{proof}
        \leavevmode \\
        The order of $[\frac{9}{4}]$ in $\mathbb{Q}/Z$ is the smallest positive integer $n$ such that $n \cdot [\frac{9}{4}] = [0]$ in $\mathbb{Q}/Z$. This means:
        \[
        n \cdot \left( \frac{9}{4} + Z \right) = 0 + Z \implies \frac{9n}{4} \in Z.
        \]
        Since $Z$ consists of rational numbers of the form $\frac{k}{1}$ for $k \in \mathbb{Z}$, $\frac{9n}{4}$ must be an integer. The smallest $n$ for which $\frac{9n}{4}$ is an integer is $4$. Therefore, the order of $[\frac{9}{4}]$ in $\mathbb{Q}/Z$ is $4$.
    \end{proof}
    \item[(ii)] Determine the order of \( ab \) in \( \mathbb{Q}/Z \), where \( a \in Z \), \( b \in \mathbb{N} \setminus \{0\} \), and \( \gcd(a, b) = 1 \).
    \begin{proof} \leavevmode \\
        To determine the order of \( [ab] \) in \( \mathbb{Q}/Z \), we have \( a \in Z \) and \( b \in \mathbb{N} \setminus \{0\} \) with \( \gcd(a, b) = 1 \). The element \( [ab] \) is defined as \( ab + Z \).

    The order of \( [ab] \) is the smallest positive integer \( n \) such that:
    \[
    n \cdot [ab] = [0] \quad \text{in } \mathbb{Q}/Z.
    \]
    This means:
    \[
    n \cdot (ab + Z) = 0 + Z \implies nab \in Z.
    \]
    Since \( Z \) consists of rational numbers of the form \( \frac{k}{1} \) for \( k \in \mathbb{Z} \), \( nab \) must be an integer.

    Given \( \gcd(a, b) = 1 \), the smallest \( n \) for which \( nab \) is an integer is \( b \). Therefore, the order of \( [ab] \) in \( \mathbb{Q}/Z \) is \( b \).

    Thus, every element in \( \mathbb{Q}/Z \) has finite order, and there are elements in \( \mathbb{Q}/Z \) of arbitrary large order.

    \end{proof}
    
    \item[(iii)] Show that \( \mathbb{Q}/Z \) is an infinite group that is not cyclic.
    \begin{proof}
        To show that \( \mathbb{Q}/Z \) is infinite, consider the elements of the form \( \left[ \frac{1}{n} \right] \) for \( n \in \mathbb{N} \). Each \( \left[ \frac{1}{n} \right] \) is distinct in \( \mathbb{Q}/Z \) because:
    \[
    \left[ \frac{1}{n} \right] = \left[ \frac{1}{m} \right] \implies \frac{1}{n} - \frac{1}{m} \in Z \implies \frac{m - n}{mn} \in \mathbb{Z},
    \]
    which is not possible unless \( n = m \). Thus, there are infinitely many distinct elements in \( \mathbb{Q}/Z \).

    To show that \( \mathbb{Q}/Z \) is not cyclic, assume for contradiction that \( \mathbb{Q}/Z \) is cyclic. Then there exists some \( [q] \) such that every element can be expressed as \( n \cdot [q] \) for some integer \( n \).

    However, for \( q = \frac{1}{p} \) (where \( p \) is a prime), the elements \( \left[ \frac{1}{p} \right] \) generate \( \mathbb{Z}/Z \), which does not include elements like \( \left[ \frac{1}{2} \right] \) if \( p \neq 2 \). Hence, there are elements in \( \mathbb{Q}/Z \) that cannot be generated by any single element, proving that \( \mathbb{Q}/Z \) is not cyclic.
    \end{proof}
    
\end{enumerate}

% ------------------------------------------------------------------------------
\section*{Exercise 34}
Prove that $(\mathbb{Q} /\{0\}, \cdot)$ is not cyclic a group.
\begin{proof}
    Assume for contradiction that \((\mathbb{Q} \setminus \{0\}, \cdot)\) is cyclic and let \( g \in \mathbb{Q} \setminus \{0\} \) be a generator. 



We can express \( g \) in its lowest terms:

\[
g = \frac{a}{b},
\]

where \( a, b \in \mathbb{Z} \setminus \{0\} \) and \( \gcd(a, b) = 1 \).

Now, consider \( g^n \):

\[
g^n = \left(\frac{a}{b}\right)^n = \frac{a^n}{b^n}.
\]

This shows that all powers of \( g \) will also be rational numbers of the form \( \frac{m}{n} \) where \( m = a^n \) and \( n = b^n \).


For \((\mathbb{Q} \setminus \{0\}, \cdot)\) to be cyclic, it must be able to generate all non-zero rational numbers, which can be represented in the form \( \frac{p}{q} \), where \( p, q \in \mathbb{Z} \setminus \{0\} \). 

1. Choose \( p \) such that \( p \) is not divisible by \( a \) or \( b \).
2. For \( g^n \) to equal \( \frac{p}{q} \), we need:
   \[
   \frac{a^n}{b^n} = \frac{p}{q},
   \]
   implying \( a^n q = p b^n \).

This requires the ability to represent every \( p \) and \( q \) using the integer powers of \( a \) and \( b \).

However, it is clear that for any fixed \( g = \frac{a}{b} \), the set of numbers \( g^n \) will produce only those rational numbers whose numerators and denominators are powers of \( a \) and \( b \), respectively. 



Since there are infinitely many rational numbers that cannot be expressed in the form \( g^n \) for any integer \( n \)  we conclude that \((\mathbb{Q} \setminus \{0\}, \cdot)\) cannot be generated by a single element.

Thus, \((\mathbb{Q} \setminus \{0\}, \cdot)\) is not a cyclic group.

    \end{proof}



% ------------------------------------------------------------------------------
\section*{Exercise 35}
35. Give an example of a non-cyclic group of order 8.
\begin{proof}
    an example of a non-cyclic group of order 8 is the dihedral group \( D_4 \). The dihedral group \( D_4 \) is the group of symmetries of a square. It consists of 8 elements: 4 rotations and 4 reflections. The group operation is composition of symmetries. The group is non-cyclic because it does not have an element of order 8.

\end{proof}

\section*{Exercise 36}
Let \( G \) be a finite group of order \( N \). Let \( \psi(d) \) be the number of elements in \( G \) of order \( d \).

\begin{enumerate}
    \item[(i)] Prove that \( \psi(d) = 0 \) if \( d \nmid N \) and that \( G \) is cyclic if and only if \( \psi(N) > 0 \).
    
    \item[(ii)] Prove that 
    \[
    \sum_{d \mid N} \psi(d) = N.
    \]

    \item[(iii)] Suppose that for every divisor \( d \) of \( N \), there is a unique subgroup \( H \) in \( G \) of order \( d \). Prove that \( \psi(d) \leq \varphi(d) \) and that \( G \) is a cyclic group.
\end{enumerate}
\begin{proof}
    \begin{enumerate}
        \item If \( d \nmid N \), by Lagrange's theorem, the order of any element in \( G \) must divide the order of the group \( N \). Therefore, there cannot be any elements of order \( d \), which implies:
        \[
        \psi(d) = 0.
        \]
    
        For the second part, \( G \) is cyclic if and only if there exists an element \( g \in G \) such that the order of \( g \) is equal to \( N \). This means that there is at least one element of order \( N \). Thus, if \( G \) is cyclic, \( \psi(N) > 0 \). Conversely, if \( \psi(N) > 0 \), then there exists at least one element of order \( N \), which generates \( G \), making \( G \) cyclic.

        \item Each element of \( G \) has a well-defined order \( d \), and by the class equation, each element of order \( d \) contributes to \( \psi(d) \) for each divisor \( d \) of \( N \). The elements of order \( d \) can be grouped according to their orders. Since the order of each element divides \( N \), the total number of elements, summed over all divisors of \( N \), must equal \( N \):
        \[
        \sum_{d \mid N} \psi(d) = N.
        \]
\item      If there is a unique subgroup \( H \) of order \( d \), then by the properties of groups, all elements in \( H \) must have the same order \( d \) (or orders that divide \( d \)). Specifically, if \( g \) is a generator of \( H \), then all elements of \( H \) can be expressed as \( g^k \) for \( k = 0, 1, \ldots, d-1 \). 

Since \( H \) has \( \varphi(d) \) elements of order \( d \) (where \( \varphi \) is the Euler's totient function), it follows that:
\[
\psi(d) \leq \varphi(d).
\]

Furthermore, since there is a unique subgroup for each divisor \( d \) of \( N \), the presence of an element of order \( N \) guarantees that \( G \) is cyclic. Thus, if \( \psi(N) > 0 \), it implies \( G \) is cyclic because it can be generated by a single element of order \( N \).

    \end{enumerate}
\end{proof}
\end{document}
