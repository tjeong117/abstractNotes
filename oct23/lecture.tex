\documentclass{article}
\usepackage{amsmath, amsthm, amssymb, amsfonts}
\usepackage{thmtools}
\usepackage{graphicx}
\usepackage{setspace}
\usepackage{geometry}
\usepackage{float}
\usepackage{hyperref}
\usepackage[utf8]{inputenc}
\usepackage[english]{babel}
\usepackage{framed}
\usepackage[dvipsnames]{xcolor}
\usepackage{tcolorbox}
\usepackage{amsmath}
\usepackage{array}
\usepackage{tikz} 
\usepackage{multirow}
\usepackage{tcolorbox}
\usepackage{xcolor}

% Define a new color for the example box



\colorlet{LightGray}{White!90!Periwinkle}
\colorlet{LightOrange}{Orange!15}
\colorlet{LightGreen}{Green!15}

\newcommand{\HRule}[1]{\rule{\linewidth}{#1}}

\colorlet{LightGray}{black!10}
\colorlet{LightOrange}{orange!15}
\colorlet{LightGreen}{green!15}
\colorlet{LightBlue}{blue!15}
\colorlet{LightCyan}{cyan!15}



\declaretheoremstyle[name=Theorem,]{thmsty}
\declaretheorem[style=thmsty,numberwithin=section]{theorem}
\usepackage{tcolorbox} % Add missing package
\tcolorboxenvironment{theorem}{colback=LightGray}

\declaretheoremstyle[name=Definition,]{thmsty}
\declaretheorem[style=thmsty,numberwithin=section]{definition}
\tcolorboxenvironment{definition}{colback=LightBlue}

\declaretheoremstyle[name=Proposition,]{prosty}
\declaretheorem[style=prosty,numberlike=theorem]{proposition}
\tcolorboxenvironment{proposition}{colback=LightOrange}


\declaretheoremstyle[name=Example,]{prosty}
\declaretheorem[style=prosty,numberlike=theorem]{example}
\tcolorboxenvironment{example}{colback=LightOrange}

\declaretheoremstyle[name=Axiom,]{prcpsty}
\declaretheorem[style=prcpsty,numberlike=theorem]{axiom}
\tcolorboxenvironment{axiom}{colback=LightGreen}

\declaretheoremstyle[name=Lemma,]{prcpsty}
\declaretheorem[style=prcpsty,numberlike=theorem]{lemma}
\tcolorboxenvironment{lemma}{colback=LightCyan}





\setstretch{1.2}
\geometry{
    textheight=9in,
    textwidth=5.5in,
    top=1in,
    headheight=12pt,
    headsep=25pt,
    footskip=30pt
}

% ------------------------------------------------------------------------------

\begin{document}

% ------------------------------------------------------------------------------
% Cover Page and ToC
% ------------------------------------------------------------------------------

\title{ \normalsize \textsc{}
		\\ [2.0cm]
		\HRule{1.5pt} \\
		\LARGE \textbf{\uppercase{RINGS}}
		\HRule{2.0pt} \\ [0.6cm] \LARGE{Cosets}
		}

\date{\today}
\author{\textbf{Author} \\ 
		Tom Jeong
        }

\maketitle
\newpage

\tableofcontents
\newpage

% ------------------------------------------------------------------------------
\section{RINGS}
Units in $\mathbb{Z} / 6\mathbb{Z}$ are [1], [5] so $\mathbb{Z} / 6\mathbb{Z}^*$
\begin{definition}
    Let $R$ Be a ring. IF $R^* = R \backslash \{0\}$ Then R is a field. 
    In other words, R is a field if every non zero element in R has a multiplicative inverse.
\end{definition}

\begin{definition}
     IOf $K \subseteq L $ are fields and K is a subring of L, then K is a subfield of L and L is an \underbar{extension} field of K. 
\end{definition}
example: $\mathbb{Q}$ is a subfield of $\mathbb{R}$ and $\mathbb{R}$ is an extensino field of $\mathbb{Q}$  \\ 
Ex. $\mathbb{R}$ is a subfield of $\mathbb{C}$ and $\mathbb{C}$ is an extension field of $\mathbb{R}$ 

\begin{definition}
    A \underline{Domain} is a ring $R$ with no zero divisors. 

\end{definition}
non ex: $\mathbb{Z} / 6 \mathbb{Z}$ is not a domain.

\begin{proposition}[3.1.3]
    Let $R$ be a domain and $a, x, y \in R$ if $a \not = 0$ and ax = ay then x  = y 
\end{proposition}
\begin{proof}
    if $ax = ay$ then $ax + (-ay) = 0 $ and $a(x + ( -y)) = 0$ since R is a domain and $a \not = 0$ x = y 
\end{proof}
\begin{proposition}[3.1.4]
    Let F be a field then F is a domain \begin{proof}
        suppose $x, y \in F, x \not = 0$ and $xy = 0$ want to show y =0 \\ 
        Since $x \not = 0$, $\exists x^{-1} \in F $ since $F$ is a field. Then $0 = x^{-1} \times 0 = x^\{-1\} (xy) = (x^\{-1\}x)y = y $

    \end{proof}
 \end{proposition}

 Ex: define $\mathbb{Q}(i) = \{a + bi | a, b \in \mathbb{Q} \} \subseteq \mathbb{C}$ 
 \\ 
 Q: Is Q(i) a field ? 
 Need to show that every nonzero element has a multiplicative inverse. 
\end{document}





 