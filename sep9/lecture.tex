\documentclass{article}
\usepackage{amsmath, amsthm, amssymb, amsfonts}
\usepackage{thmtools}
\usepackage{graphicx}
\usepackage{setspace}
\usepackage{geometry}
\usepackage{float}
\usepackage{hyperref}
\usepackage[utf8]{inputenc}
\usepackage[english]{babel}
\usepackage{framed}
\usepackage[dvipsnames]{xcolor}
\usepackage{tcolorbox}
\usepackage{amsmath}
\usepackage{array}
\usepackage{multirow}


\colorlet{LightGray}{White!90!Periwinkle}
\colorlet{LightOrange}{Orange!15}
\colorlet{LightGreen}{Green!15}

\newcommand{\HRule}[1]{\rule{\linewidth}{#1}}

\colorlet{LightGray}{black!10}
\colorlet{LightOrange}{orange!15}
\colorlet{LightGreen}{green!15}
\colorlet{LightBlue}{blue!15}
\colorlet{LightCyan}{cyan!15}



\declaretheoremstyle[name=Theorem,]{thmsty}
\declaretheorem[style=thmsty,numberwithin=section]{theorem}
\usepackage{tcolorbox} % Add missing package
\tcolorboxenvironment{theorem}{colback=LightGray}

\declaretheoremstyle[name=Definition,]{thmsty}
\declaretheorem[style=thmsty,numberwithin=section]{definition}
\tcolorboxenvironment{definition}{colback=LightBlue}

\declaretheoremstyle[name=Proposition,]{prosty}
\declaretheorem[style=prosty,numberlike=theorem]{proposition}
\tcolorboxenvironment{proposition}{colback=LightOrange}


\declaretheoremstyle[name=Proof,]{prosty}
\declaretheorem[style=prosty,numberlike=theorem]{proofbox}
\tcolorboxenvironment{proofbox}{colback=LightOrange}

\declaretheoremstyle[name=Axiom,]{prcpsty}
\declaretheorem[style=prcpsty,numberlike=theorem]{axiom}
\tcolorboxenvironment{axiom}{colback=LightGreen}

\declaretheoremstyle[name=Lemma,]{prcpsty}
\declaretheorem[style=prcpsty,numberlike=theorem]{lemma}
\tcolorboxenvironment{lemma}{colback=LightCyan}





\setstretch{1.2}
\geometry{
    textheight=9in,
    textwidth=5.5in,
    top=1in,
    headheight=12pt,
    headsep=25pt,
    footskip=30pt
}

% ------------------------------------------------------------------------------

\begin{document}

% ------------------------------------------------------------------------------
% Cover Page and ToC
% ------------------------------------------------------------------------------

\title{ \normalsize \textsc{}
		\\ [2.0cm]
		\HRule{1.5pt} \\
		\LARGE \textbf{\uppercase{Lecture 6 sep 9}}
		\HRule{2.0pt} \\ [0.6cm] \LARGE{Cosets}
		}

\date{\today}
\author{\textbf{Author} \\ 
		Tom Jeong
        }

\maketitle
\newpage

\tableofcontents
\newpage

% ------------------------------------------------------------------------------
\section{Normal Subgroup}
\begin{proposition} 
 If $N$ is a normal subgroup then $(xN) (yN) = (xy) N \forall x,y \in G$ 
 \end{proposition}
 Corollary 2.3.3 Let $N$ be a normal subgroup of G then composition of left cosets makes $G / N $ into a group weher $(xN) (yN) = (xy)N \forall xN, yN \in G / N $
 \begin{proof}
     Since cmposition in G is associative, $(g_1N) (g_2N) (g_3N) = g_1N (g_2N g_3N) = g_1N (g_2 g_3) N = (g_1 g_2) g_3 N = (g_1 g_2) N (g_3 N) = (g_1 N g_2 N) (g_3 N)$
     The identity is $N = eN$ since $eN gN = (eg) N = gN $ and $gN eN = gN$. The inverse of $gN$ is $g^{-1} N$ since $gN g^{-1} N = (gg^{-1}) N = N$ and $g^{-1} N gN = (g^{-1} g) N = N$
    
 \end{proof}
\begin{definition}
    Let $N$ be a normal subgroup of $G$. The group $G / N $ (g mod n) is called the Quotient Group  of G by N. 
\end{definition}
Let us look at some examples... $G = (\mathbb{Z} / 6\mathbb{Z}, +)$ and $N = \{[0], [3]\}$  left cosets of N: 
\begin{enumerate}
    \item $[0] + N = \{[0], [3]\}$
    \item $[1] + N = \{[1], [4]\}$
    \item $[2] + N = \{[2], [5]\}$
    \end {enumerate}
    Then $ G / N = \mathbb{Z} / 6\mathbb{Z} / \{[0], [3]\} = \{ N, [1] + N, [2] + N\}$


    \begin{definition} \leavevmode \\ 
        $(\mathbb{Z} / n\mathbb{Z})^{*} = \{ [a] \in \mathbb{Z }/ n \mathbb{Z} | gcd(a,n) = 1\}$  
    \end{definition}
    example: $(\mathbb{Z} / 8 \mathbb{Z})^{*} = \{ [1], [3], [5], [7]\}$

        
    \begin{table}[h]
    \centering
    \begin{tabular}{|c|c|c|c|c|}
    \hline
    $\cdot$ & $[1]$ & $[3]$ & $[5]$ & $[7]$ \\
    \hline
    $[1]$ & $[1]$ & $[3]$ & $[5]$ & $[7]$ \\
    \hline
    $[3]$ & $[3]$ & $[1]$ & $[7]$ & $[5]$ \\
    \hline
    $[5]$ & $[5]$ & $[7]$ & $[1]$ & $[3]$ \\
    \hline
    $[7]$ & $[7]$ & $[5]$ & $[3]$ & $[1]$ \\
    \hline
    \end{tabular}
    \caption{Multiplication Table for $(\mathbb{Z} / 8 \mathbb{Z})^{*} = \{ [1], [3], [5], [7]\}$}
    \end{table}
    \underline{NOTE:} If $p$ is prime, $(\mathbb{Z} / p \mathbb{Z})^{*} = \{[0], [1], \dots, [p-1]\}$
\section{Group homomorphism}
    \begin{definition}[2.4.1] \leavevmode \\
        Let $G$ and $k$ be groups. A function $f: G \rightarrow K$ is called a group homomorphism if $f(ab) = f(a) f(b) \forall a,b \in G$
    \end{definition}


    examples: $f: (\mathbb{Z}, +) \rightarrow (\mathbb{Z}, +)$ 
    $$x \rightarrowtail 2x$$
    \begin{align*}
        f(x+y) &= 2(x+y) \\ 
        &= 2x + 2y \\
        &= f(x) + f(y) 
    \end{align*}
    $\uparrow$ : Not an isomorphisim because it is not surjective. It is injective though.  \\


    example 2: $f: (\mathbb{R}, +) \rightarrow (\mathbb{R}^*, \cdot)$
    \begin{align*}
        x &\rightarrowtail e^x \\ 
        f(x+y) &= e^{x+y} \\
        &= e^x e^y \\
        &= f(x) f(y)
    \end{align*}
    $\uparrow$ : Not an isomorphisim because it is not surjective (the image of $e^x$ is always greater or equal to 0). It is injective though.  \\

    
    example3: determinant $f: (GL_2(\mathbb{R}), \cdot) \rightarrow (\mathbb{R}^*, \cdot)$ 
    \begin{align*}
        A &\rightarrowtail det(A) \\ 
        f(AB) &= det(AB) \\
        &= det(A) det(B) \\
        &= f(A) f(B)
    \end{align*}
    \begin{definition}[2.4.5] \leavevmode \\
        let $f: G \rightarrow K$ be a group homomorphism. 
        \begin{enumerate}
            \item The kernel of $f$ is the set is $ker(f) = \{ g \in G | f(g) = e_K \}$
            \item the image of $f$ is the set $Im(f) = \{ f(g) | g \in G \}$
            \item \underline{if f is a bijection, then $f$ is called an isomorphism} and we say $G$ and $K$ are isomorphic and write $G \cong K$
        \end{enumerate} 
    \end{definition}
example: recall $D_3 = \{e, r_1, r_2, s_1, s_2, s_3\}$ aka symmetries of a triangle 
and $S_3 = \{e, (1,2), (1,3), (2,3), (1,2,3), (1,3,2)\}$ aka permutations of 3 elements.
exercise: shwo that $D_3 \cong S_3$ by explicitly constructing an isomorphism $\phi: D_3 \rightarrow S_3$
\\ 
\begin{align*}
    \phi(s_1) &=  \begin{pmatrix} 
        1 & 2 & 3 \\ 
        1 & 3 & 2
        \end{pmatrix}
\end{align*}
ex. Let $N \trianglelefteq G$ then the function $\Pi: G \rightarrow G / N$ given by $f(g) = gN$ is a surjective group homomorphism with kernel $N$
\begin{proof}
    $\Pi(gh) = (gh) N = gN hN = \Pi(g) \Pi(h)$ \\ 
    $Ker(\Pi) = \{ g \in G | \Pi(g) = N \} = \{ g \in G | gN = N \} = N$ \\ 
    $Im(\pi) = \{ gN | g \in G \} = G / N$ \\ \\ 
    why is the identity of $G / N$ $N$? 
    \begin{align*}
        \Pi(e) &= eN = N \\ 
        \Pi(g) &= gN = N \rightarrow g \in N
    \end{align*}
\end{proof}

\begin{proposition}[2.4.9]
    Let $f: G \rightarrow K$ be a group homomorphism. 
    \begin{enumerate}
        \item the image $f(G) \subseteq K$ is a subgroup kf $K$
        \item The kernel $ker(f) \subseteq G$ and $ker(f) \trianglelefteq G$ it is a normal subgroup of $G$ 
        \item $f$ is injective iff $ker f = \{e_G\}$
    \end{enumerate}
   
\end{proposition}
\begin{proof}\leavevmode \\ 
    \begin{enumerate} 
        \item \underline{identity} $f(g) = f(ge_G) = f(g) f(e_G) \rightarrow f(e_G) = e_K$ \\ Hence $e_K \in Im(f)$  
        \\  \underline{Closure} if $k_1, k_2 \in Im(f)$ then $k_1 = f(g_1)$ and $k_2 = f(g_2)$ for some $g_1, g_2 \in G$ then $k_1 k_2 = f(g_1) f(g_2) = f(g_1 g_2) \in Im(f)$
        \\ \underline{Inverse} if $k \in Im(f)$ then $k = f(g)$ for some $g \in G$. And $e = k k^{-1} = f(g) f(g)^{-1} $ and $e_K= f(e_G) = f(g g^{-1}) = f(g) f(g)^{-1} = k k^{-1} \in Im(f)$ hence $(f(g))^{-1} = f(g^{-1}) \in Im(f)$
        \item \underline{Id}: since $f(e_G) = e_K$ then $e_G \in ker(f)$ \\ 
        \underline{closure} suppose $x, y \in ker(f)$ then $f(xy) = f(x) f(y) = e_K e_K = e_K$ hence $xy \in ker(f)$ \\
        \underline{inverses} now if $x \in ker(f)$ by definition $e = f(x)$ then $f(x^{-1}) = f(x)^{-1} = e_K^{-1} = e_K$ hence $x^{-1} \in ker(f)$ \\
    \end{enumerate}
\end{proof}

 \end{document}
 