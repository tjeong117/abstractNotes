\documentclass{article}
\usepackage{amsmath, amsthm, amssymb, amsfonts}
\usepackage{thmtools}
\usepackage{graphicx}
\usepackage{setspace}
\usepackage{geometry}
\usepackage{float}
\usepackage{hyperref}
\usepackage[utf8]{inputenc}
\usepackage[english]{babel}
\usepackage{framed}
\usepackage[dvipsnames]{xcolor}
\usepackage{tcolorbox}

\colorlet{LightGray}{White!90!Periwinkle}
\colorlet{LightOrange}{Orange!15}
\colorlet{LightGreen}{Green!15}

\newcommand{\HRule}[1]{\rule{\linewidth}{#1}}

\colorlet{LightGray}{black!10}
\colorlet{LightOrange}{orange!15}
\colorlet{LightGreen}{green!15}
\colorlet{LightBlue}{blue!15}
\colorlet{LightCyan}{cyan!15}



\declaretheoremstyle[name=Theorem,]{thmsty}
\declaretheorem[style=thmsty,numberwithin=section]{theorem}
\usepackage{tcolorbox} % Add missing package
\tcolorboxenvironment{theorem}{colback=LightGray}

\declaretheoremstyle[name=Solution,]{thmsty}
\declaretheorem[style=thmsty,numberwithin=section]{solution}
\tcolorboxenvironment{solution}{colback=LightBlue}

\declaretheoremstyle[name=Proposition,]{prosty}
\declaretheorem[style=prosty,numberlike=theorem]{proposition}
\tcolorboxenvironment{proposition}{colback=LightOrange}


\declaretheoremstyle[name=Proof,]{prosty}
\declaretheorem[style=prosty,numberlike=theorem]{proofbox}
\tcolorboxenvironment{proofbox}{colback=LightOrange}

\declaretheoremstyle[name=Axiom,]{prcpsty}
\declaretheorem[style=prcpsty,numberlike=theorem]{axiom}
\tcolorboxenvironment{axiom}{colback=LightGreen}

\declaretheoremstyle[name=Lemma,]{prcpsty}
\declaretheorem[style=prcpsty,numberlike=theorem]{lemma}
\tcolorboxenvironment{lemma}{colback=LightCyan}





\setstretch{1.2}
\geometry{
    textheight=9in,
    textwidth=5.5in,
    top=1in,
    headheight=12pt,
    headsep=25pt,
    footskip=30pt
}

% ------------------------------------------------------------------------------

\begin{document}

% ------------------------------------------------------------------------------
% Cover Page and ToC
% ------------------------------------------------------------------------------

\title{ \normalsize \textsc{}
		\\ [2.0cm]
		\HRule{1.5pt} \\
		\LARGE \textbf{\uppercase{Lecture 1 }}
		\HRule{2.0pt} \\ [0.6cm] \LARGE{}
		}

\date{\today}
\author{\textbf{Author} \\ 
		Tom Jeong
        }



% ------------------------------------------------------------------------------
%: 2, 3, 6 i-v, 8, 11
\section{1.2}
Let $x, d \in \mathbb{Z} $, where $d > 0$. Prove that $M \cap N \neq \emptyset$, where $M = \{x - qd | q \in \mathbb{Z}\}$.
\begin{solution}
    Let $x, d \in \mathbb{Z}$, where $d > 0$. 
Let $M = \{x - qd \mid q \in \mathbb{Z}\}$ and $N = \{x + pd \mid p \in \mathbb{Z}\}$.

Consider $x \in \mathbb{Z}$:
\begin{align*}
x &= x - 0d \in M \quad (\text{where } q = 0) \\
x &= x + 0d \in N \quad (\text{where } p = 0)
\end{align*}

Therefore, $x \in M \cap N$, and thus $M \cap N \neq \emptyset$.
\end{solution}
\section{1.3}
Let $a,b, N \in \mathbb{Z}$ where $N > 0$ Prove that $[a][b] = [[a][b]]$ where $[x]$ denotes the remainder of $x$ after divison by $N$ 
\begin{solution}
    By definition of modular arithmetic:
\begin{align*}
a &\equiv [a] \pmod{N} \\
b &\equiv [b] \pmod{N}
\end{align*}

This means there exist integers $k$ and $m$ such that:
\begin{align*}
a &= [a] + kN \\
b &= [b] + mN
\end{align*}

Multiplying these equations:
\begin{align*}
ab &= ([a] + kN)([b] + mN) \\
   &= [a][b] + [a]mN + [b]kN + kmN^2
\end{align*}

Taking both sides modulo $N$:
\begin{align*}
[ab] &\equiv [[a][b] + [a]mN + [b]kN + kmN^2] \pmod{N} \\
     &\equiv [[a][b]] \pmod{N}
\end{align*}
and thus $[ab] = [[a][b]]$.

this is because $[a]mN$, $[b]kN$, and $kmN^2$ are all multiples of $N$.

\end{solution}


\section{1.6}
\begin{solution}
    Let $a = a_0 \cdot 10^0 + a_1 \cdot 10^1 + a_2 \cdot 10^2 + \cdots + a_n \cdot 10^n$, where $0 \leq a_i < 10$.

(i) 2 divides $a$ if and only if 2 divides $a_0$:

$a \equiv a_0 \cdot 10^0 + a_1 \cdot 10^1 + a_2 \cdot 10^2 + \cdots + a_n \cdot 10^n \pmod{2}$
$\equiv a_0 + 0 + 0 + \cdots + 0 \pmod{2}$ (since $10^k \equiv 0 \pmod{2}$ for $k \geq 1$)
$\equiv a_0 \pmod{2}$

Therefore, $a \equiv 0 \pmod{2}$ if and only if $a_0 \equiv 0 \pmod{2}$. \\

(ii) 4 divides $a$ if and only if 4 divides $a_0 + 2a_1$:

$a \equiv a_0 \cdot 10^0 + a_1 \cdot 10^1 + a_2 \cdot 10^2 + \cdots + a_n \cdot 10^n \pmod{4}$
$\equiv a_0 + 2a_1 + 0 + \cdots + 0 \pmod{4}$ (since $10 \equiv 2 \pmod{4}$ and $10^k \equiv 0 \pmod{4}$ for $k \geq 2$)
$\equiv a_0 + 2a_1 \pmod{4}$

Therefore, $a \equiv 0 \pmod{4}$ if and only if $a_0 + 2a_1 \equiv 0 \pmod{4}$. \\ 

(iii) 8 divides $a$ if and only if 8 divides $a_0 + 2a_1 + 4a_2$:

$a \equiv a_0 \cdot 10^0 + a_1 \cdot 10^1 + a_2 \cdot 10^2 + \cdots + a_n \cdot 10^n \pmod{8}$
$\equiv a_0 + 2a_1 + 4a_2 + 0 + \cdots + 0 \pmod{8}$ (since $10 \equiv 2 \pmod{8}$, $10^2 \equiv 4 \pmod{8}$, and $10^k \equiv 0 \pmod{8}$ for $k \geq 3$)
$\equiv a_0 + 2a_1 + 4a_2 \pmod{8}$

Therefore, $a \equiv 0 \pmod{8}$ if and only if $a_0 + 2a_1 + 4a_2 \equiv 0 \pmod{8}$ \\

(iv) 5 divides $a$ if and only if 5 divides $a_0$:

$a \equiv a_0 \cdot 10^0 + a_1 \cdot 10^1 + a_2 \cdot 10^2 + \cdots + a_n \cdot 10^n \pmod{5}$
$\equiv a_0 + 0 + 0 + \cdots + 0 \pmod{5}$ (since $10^k \equiv 0 \pmod{5}$ for $k \geq 1$)
$\equiv a_0 \pmod{5}$

Therefore, $a \equiv 0 \pmod{5}$ if and only if $a_0 \equiv 0 \pmod{5}$.

(v) Let $a = a_0 \cdot 10^0 + a_1 \cdot 10^1 + a_2 \cdot 10^2 + \cdots + a_n \cdot 10^n$, where $0 \leq a_i < 10$.

9 divides $a$ if and only if 9 divides the sum $a_0 + a_1 + \cdots + a_n$ of its digits:

$a \equiv a_0 \cdot 10^0 + a_1 \cdot 10^1 + a_2 \cdot 10^2 + \cdots + a_n \cdot 10^n \pmod{9}$
$\equiv a_0 + a_1 + a_2 + \cdots + a_n \pmod{9}$ (since $10^k \equiv 1 \pmod{9}$ for all $k \geq 0$)

Therefore, $a \equiv 0 \pmod{9}$ if and only if $(a_0 + a_1 + \cdots + a_n) \equiv 0 \pmod{9}$.

Note: $10^k \equiv 1 \pmod{9}$ for all $k \geq 0$ because $10^k - 1 = (10-1)(10^{k-1} + 10^{k-2} + \cdots + 1)$ is divisible by 9 for $k \geq 1$, and $10^0 - 1 = 0$ is also divisible by 9.

\end{solution}
\section{1.8}
$3 | 4^n -1$ for all $n \in \mathbb{N} $
\begin{solution}
    e will prove that $3 \mid 4^n - 1$ for all $n \in \mathbb{N}$ using mathematical induction.

Base case: For $n = 1$,
$4^1 - 1 = 3$, which is clearly divisible by 3.

Inductive step: Assume the statement holds for some $k \in \mathbb{N}$, i.e., $3 \mid 4^k - 1$.
This means there exists an integer $m$ such that $4^k - 1 = 3m$.

Now, let's prove it holds for $k+1$:

\begin{align*}
4^{k+1} - 1 &= 4 \cdot 4^k - 1 \\
&= 4(4^k - 1) + 4 - 1 \\
&= 4(3m) + 3 \quad \text{(substituting $4^k - 1 = 3m$)} \\
&= 12m + 3 \\
&= 3(4m + 1)
\end{align*}

Since $4m + 1$ is an integer, we have shown that $4^{k+1} - 1$ is divisible by 3.
\\ we can also use the fact that $ 4 \equiv 1 \pmod{3} $ and any power to the 4 bigger than 1 would be divisible by 3.
\end{solution} 

\section{1.11}
\begin{solution}
    et $x$, $y$, $z$, $d \in \mathbb{Z}$ where $d \neq 0$.

(i) Reflexivity: $x \equiv x \pmod{d}$

By definition, $x \equiv x \pmod{d}$ if $d \mid (x - x)$.
$x - x = 0 = d \cdot 0$
Therefore, $d \mid (x - x)$, so $x \equiv x \pmod{d}$.

(ii) Symmetry: If $x \equiv y \pmod{d}$, then $y \equiv x \pmod{d}$

Given: $x \equiv y \pmod{d}$
This means $d \mid (x - y)$, so there exists an integer $k$ such that $x - y = dk$
Rearranging: $y - x = -dk = d(-k)$
Since $-k$ is an integer, $d \mid (y - x)$
Therefore, $y \equiv x \pmod{d}$

(iii) Transitivity: If $x \equiv y \pmod{d}$ and $y \equiv z \pmod{d}$, then $x \equiv z \pmod{d}$

Given: $x \equiv y \pmod{d}$ and $y \equiv z \pmod{d}$
This means there exist integers $k$ and $m$ such that:
$x - y = dk$ and $y - z = dm$
Adding these equations:
$(x - y) + (y - z) = dk + dm$
$x - z = d(k + m)$
Since $k + m$ is an integer, $d \mid (x - z)$
Therefore, $x \equiv z \pmod{d}$

(iv) 
\end{solution}
% ------------------------------------------------------------------------------

\section{2.1}
\begin{solution}
    \begin{enumerate}
        \item Injectivity: \\ Let $x_1, x_2 \in G$ such that $\xi(x_1) = \xi(x_2)$.
        Then $x_1g = x_2g$.
        Multiplying both sides by $g^{-1}$ on the right (which exists because G is a group):
        $x_1g g^{-1} = x_2g g^{-1}$
        $x_1 = x_2$ (by the properties of inverse elements in a group)
        
        Thus, if $\xi(x_1) = \xi(x_2)$, then $x_1 = x_2$, proving that $\xi$ is injective.
        \item Surjectivity: \\ Let $y \in G$ be arbitrary. We need to find an $x \in G$ such that $\xi(x) = y$.
        Consider $x = yg^{-1}$.
        Then $\xi(x) = \xi(yg^{-1}) = yg^{-1}g = y$ (by the properties of inverse elements)
        
        Thus, for any $y \in G$, we can find an $x \in G$ (namely, $yg^{-1}$) such that $\xi(x) = y$, proving that $\xi$ is surjective.
     
     Since $\xi$ is both injective and surjective, we conclude that $\xi$ is bijective.
    \end{enumerate}
\end{solution}
\end{document}
