\documentclass{article}
\usepackage{amsmath, amsthm, amssymb, amsfonts}
\usepackage{thmtools}
\usepackage{graphicx}
\usepackage{setspace}
\usepackage{geometry}
\usepackage{float}
\usepackage{hyperref}
\usepackage[utf8]{inputenc}
\usepackage[english]{babel}
\usepackage{framed}
\usepackage[dvipsnames]{xcolor}
\usepackage{tcolorbox}

\colorlet{LightGray}{White!90!Periwinkle}
\colorlet{LightOrange}{Orange!15}
\colorlet{LightGreen}{Green!15}

\newcommand{\HRule}[1]{\rule{\linewidth}{#1}}

\colorlet{LightGray}{black!10}
\colorlet{LightOrange}{orange!15}
\colorlet{LightGreen}{green!15}
\colorlet{LightBlue}{blue!15}
\colorlet{LightCyan}{cyan!15}



\declaretheoremstyle[name=Theorem,]{thmsty}
\declaretheorem[style=thmsty,numberwithin=section]{theorem}
\usepackage{tcolorbox} % Add missing package
\tcolorboxenvironment{theorem}{colback=LightGray}

\declaretheoremstyle[name=Definition,]{thmsty}
\declaretheorem[style=thmsty,numberwithin=section]{definition}
\tcolorboxenvironment{definition}{colback=LightBlue}

\declaretheoremstyle[name=Proposition,]{prosty}
\declaretheorem[style=prosty,numberlike=theorem]{proposition}
\tcolorboxenvironment{proposition}{colback=LightOrange}


\declaretheoremstyle[name=Proof,]{prosty}
\declaretheorem[style=prosty,numberlike=theorem]{proofbox}
\tcolorboxenvironment{proofbox}{colback=LightOrange}

\declaretheoremstyle[name=Axiom,]{prcpsty}
\declaretheorem[style=prcpsty,numberlike=theorem]{axiom}
\tcolorboxenvironment{axiom}{colback=LightGreen}

\declaretheoremstyle[name=Lemma,]{prcpsty}
\declaretheorem[style=prcpsty,numberlike=theorem]{lemma}
\tcolorboxenvironment{lemma}{colback=LightCyan}





\setstretch{1.2}
\geometry{
    textheight=9in,
    textwidth=5.5in,
    top=1in,
    headheight=12pt,
    headsep=25pt,
    footskip=30pt
}

% ------------------------------------------------------------------------------

\begin{document}

% ------------------------------------------------------------------------------
% Cover Page and ToC
% ------------------------------------------------------------------------------
% Exercises 19, 20, 21, 24, 26, 28, 31, 32

\newpage

% ------------------------------------------------------------------------------
\section*{Exercise 19} 
\begin{enumerate}
    \item Compute the inverse of $[3]$ in $(\mathbb{Z}/8\mathbb{Z})^*$.
    \begin{proof}
        \leavevmode \\ 
        We have that $[3] \in (\mathbb{Z}/8\mathbb{Z})^*$ if and only if $\gcd(3, 8) = 1$. Since $\gcd(3, 8) = 1$, we have that $[3]$ is invertible in $(\mathbb{Z}/8\mathbb{Z})^*$. We can find the inverse of $[3]$ by solving the following equation:
        \begin{align*}
            [3]x &\equiv [1] \pmod{8} \\
            3x &\equiv 1 \pmod{8} \\
            x &\equiv 3^{-1} \pmod{8}
        \end{align*}
        We can find the inverse of $[3]$ by using the Extended Euclidean Algorithm. We have that:
        \begin{align*}
            8 &= 3(2) + 2 \\
            3 &= 2(1) + 1
        \end{align*}
        We can now find the inverse of $[3]$ by working backwards:
        \begin{align*}
            1 &= 3 - 2(1) \\
            &= 3 - (8 - 3(2)) \\
            &= 3 - 8 + 6 \\
            &= -8 + 9 \\
            &= 1
        \end{align*}
        Therefore, the inverse of $[3]$ in $(\mathbb{Z}/8\mathbb{Z})^*$ is $[3]^{-1} = [3]$.
    \end{proof}
    \item Compute the inverse of $[5]$ in $(\mathbb{Z}/13\mathbb{Z})^*$.
    \begin{proof}
         \leavevmode \\ 
        
\[
    \gcd(5, 13) = 1
    \]
    
    Since \(5\) is invertible, we seek \(b\) such that:
    
    \[
    5b \equiv 1 \mod 13
    \]
    
    Using the Extended Euclidean Algorithm:
    
    \begin{align*}
    13 &= 2 \cdot 5 + 3 \\
    5 &= 1 \cdot 3 + 2 \\
    3 &= 1 \cdot 2 + 1 \\
    2 &= 2 \cdot 1 + 0
    \end{align*}
    
    Now, we backtrack:
    
    From the third equation:
    \[
    1 = 3 - 1 \cdot 2
    \]
    
    Substituting for \(2\):
    \[
    1 = 3 - 1 \cdot (5 - 1 \cdot 3) = 3 - 5 + 3 = 2 \cdot 3 - 5
    \]
    
    Substituting for \(3\):
    \[
    1 = 2 \cdot (13 - 2 \cdot 5) - 5 = 2 \cdot 13 - 4 \cdot 5 - 5 = 2 \cdot 13 - 5 \cdot 5
    \]
    
    This implies:
    \[
    1 \equiv -5 \cdot 5 \mod 13
    \]
    
    Thus, the inverse of \([5]\) in \((\mathbb{Z}/13\mathbb{Z})^*\) is:
    
    \[
    \boxed{[8]}
    \]
        Therefore, the inverse of $[5]$ in $(\mathbb{Z}/13\mathbb{Z})^*$ is $[5]^{-1} = [-2] = [11]$.

    \end{proof}
\end{enumerate}

\section*{Exercise 20}
Prove that the inverse map of a group isomorphism is also a group homomorphism.
\begin{proof}
    \leavevmode \\ 
    Let $G$ and $H$ be groups and let $\phi: G \to H$ be a group isomorphism. We know that $\phi$ is bijective, so it has an inverse $\phi^{-1}: H \to G$. We want to show that $\phi^{-1}$ is a group homomorphism. Let $a, b \in H$. We have that:
    \begin{align*}
        \phi^{-1}(a \cdot b) &= \phi^{-1}(\phi(\phi^{-1}(a)) \cdot \phi(\phi^{-1}(b))) \\
        &= \phi^{-1}(\phi(\phi^{-1}(a) \cdot \phi^{-1}(b))) \\
        &= \phi^{-1}(a) \cdot \phi^{-1}(b)
    \end{align*}
    Therefore, $\phi^{-1}$ is a group homomorphism.
\end{proof}

\section*{Exercise 21}
Prove that $G$ is abelian if and only if the map $f: G \to G$ given by $f(g) = g^2$ is a group homomorphism.
\begin{proof}
        \leavevmode \\  
        Let \( G \) be a group, and define the map \( f: G \to G \) by \( f(g) = g^2 \).

      
        Assume $G$ is abelian. 
        We want to show that \( f \) is a homomorphism, i.e., \( f(g_1 g_2) = f(g_1) f(g_2) \) for all \( g_1, g_2 \in G \).
        
        Calculating the left-hand side:
        \[
        f(g_1 g_2) = (g_1 g_2)^2 = g_1 g_2 g_1 g_2
        \]
        
        Since \( G \) is abelian, we can rearrange the terms:
        \[
        g_1 g_2 g_1 g_2 = g_1 g_1 g_2 g_2 = g_1^2 g_2^2
        \]
        
        Now calculating the right-hand side:
        \[
        f(g_1) f(g_2) = g_1^2 g_2^2
        \]
        
        Thus,
        \[
        f(g_1 g_2) = f(g_1) f(g_2)
        \]
        This shows that \( f \) is a homomorphism. \\ now assume $f$ is a homoomorphism
        
        We need to show that \( G \) is abelian, i.e., \( g_1 g_2 = g_2 g_1 \) for all \( g_1, g_2 \in G \).
        
        Using the homomorphism property, we have:
        \[
        f(g_1 g_2) = f(g_1) f(g_2)
        \]
        Substituting the definition of \( f \):
        \[
        (g_1 g_2)^2 = g_1^2 g_2^2
        \]
        
        Expanding the left-hand side:
        \[
        g_1 g_2 g_1 g_2 = g_1^2 g_2^2
        \]
        
        Rearranging gives:
        \[
        g_1 g_2 g_1 g_2 = g_1 g_1 g_2 g_2
        \]
        
        Cancelling \( g_1 \) from the left (since \( G \) is a group and hence has inverses), we can assume:
        \[
        g_2 g_1 = g_1 g_2
        \]
        
        This shows \( g_1 g_2 = g_2 g_1 \) for all \( g_1, g_2 \in G \), confirming that \( G \) is abelian.
    \end{proof}


    \section*{Excercise 24}
    Prove that $\begin{bmatrix}
        1 & 1 \\
        0 & 1
    \end{bmatrix} \in GL_2(\mathbb{R})$ has infinite order in the group $GL_2(\mathbb{R})$.
   \begin{proof}
        \leavevmode \\
        Let $A = \begin{bmatrix}
            1 & 1 \\
            0 & 1
        \end{bmatrix} \in GL_2(\mathbb{R})$. We want to show that $A$ has infinite order in $GL_2(\mathbb{R})$. We have that:
        \begin{align*}
            A^n &= \begin{bmatrix}
                1 & 1 \\
                0 & 1
            \end{bmatrix}^n \\
            &= \begin{bmatrix}
                1 & n \\
                0 & 1
            \end{bmatrix}
        \end{align*}
        We can see that $A^n$ is the identity matrix if and only if $n = 0$.  Since there are no positive integers \(n\) such that \(A^n = I\), we conclude that \(A\) has infinite order in \(GL_2(\mathbb{R})\).Therefore, $A$ has infinite order in $GL_2(\mathbb{R})$.
        
   \end{proof}

    \section*{Exercise 26}
    Let $G$ be an abelian group, $K$ a group, and $f: G \to K$ a group homomorphism. We want to show that $f(G) \subseteq K$ is an abelian subgroup of $K$.
    \begin{proof}
        \leavevmode \\ 
        Let \( G \) be an abelian group, \( K \) a group, and \( f: G \to K \) a group homomorphism. We aim to show that \( f(G) \) is an abelian subgroup of \( K \).

        
        To show that \( f(G) \) is a subgroup of \( K \), we need to verify that it satisfies the subgroup criteria:
        
        1. e: Since \( f \) is a homomorphism and \( e_G \) is the identity in \( G \), we have:
           \[
           f(e_G) = e_K,
           \]
           where \( e_K \) is the identity in \( K \). Thus, \( e_K \in f(G) \).
        
        2. Closures: Let \( x, y \in f(G) \). Then there exist \( g_1, g_2 \in G \) such that \( x = f(g_1) \) and \( y = f(g_2) \). Since \( f \) is a homomorphism, we have:
           \[
           xy = f(g_1) f(g_2) = f(g_1 g_2).
           \]
           Since \( g_1 g_2 \in G \), it follows that \( xy \in f(G) \).
        
        3. inverse Let \( x \in f(G) \). Then there exists \( g \in G \) such that \( x = f(g) \). The inverse of \( x \) in \( K \) is given by:
           \[
           x^{-1} = f(g)^{-1} = f(g^{-1}).
           \]
           Since \( g^{-1} \in G \), we have \( x^{-1} \in f(G) \).
        
        Since all three conditions for a subgroup are satisfied, we conclude that \( f(G) \) is a subgroup of \( K \).
        
  
        
        Let \( x, y \in f(G) \). Then there exist \( g_1, g_2 \in G \) such that \( x = f(g_1) \) and \( y = f(g_2) \). Since \( G \) is abelian, we have:
        \[
        g_1 g_2 = g_2 g_1.
        \]
        Using the homomorphism property, we get:
        \[
        xy = f(g_1) f(g_2) = f(g_1 g_2) = f(g_2 g_1) = f(g_2) f(g_1) = yx.
        \]
        Thus, \( xy = yx \), showing that \( f(G) \) is abelian.
        
        Therefore, we conclude that \( f(G) \) is an abelian subgroup of \( K \).
    \end{proof}

    \section*{Exercise 28}
     Prove that $(\mathbb{Z}/13\mathbb{Z})^*$ is a cyclic group by finding a generator.



\begin{proof}
    \leavevmode \\
    We want to show that $(\mathbb{Z}/13\mathbb{Z})^*$ is a cyclic group by finding a generator. We know that $(\mathbb{Z}/13\mathbb{Z})^*$ is the set of all elements in $\mathbb{Z}/13\mathbb{Z}$ that are relatively prime to $13$. We can find a generator for $(\mathbb{Z}/13\mathbb{Z})^*$ by finding an element of order $12$. We can find an element of order $12$ by checking the orders of the elements in $(\mathbb{Z}/13\mathbb{Z})^*$:
    \begin{align*}
        [1]^1 &= [1] \\
        [2]^1 &= [2] \\
        [3]^1 &= [3] \\
        [4]^1 &= [4] \\
        [5]^1 &= [5] \\
        [6]^2 &= [1] \\
        [7]^1 &= [7] \\
        [8]^2 &= [1] \\
        [9]^2 &= [1] \\
        [10]^2 &= [1] \\
        [11]^2 &= [1] \\
        [12]^2 &= [1]
    \end{align*}
    We can see that $[6]$ is an element of order $12$ in $(\mathbb{Z}/13\mathbb{Z})^*$. Therefore, $(\mathbb{Z}/13\mathbb{Z})^*$ is a cyclic group with generator $[6]$.
\end{proof}


\section*{Exercise 31}
31. (i) write down all the elements with order 7 in Z/28Z?
(ii) How many subgroups are there of order 7 in Z/28Z?

\begin{enumerate}
    \item write down all the elements with order 7 in $\mathbb{Z}/28\mathbb{Z}$.
    \begin{proof}
        We want to find all the elements of order \(7\) in \(\mathbb{Z}/28\mathbb{Z}\). An element \([x]\) has order \(7\) if and only if \(7x \equiv 0 \mod 28\) and \(x \not\equiv 0 \mod 28\).

        This means \(x\) must be a multiple of \(4\) (since \( \frac{28}{7} = 4\)), but not a multiple of \(28\). The candidates are \(4, 8, 12, 16, 20, 24\).

        Calculating orders:
        \begin{align*}
            [4]^7 &\equiv [0], \\
            [8]^7 &\equiv [0], \\
            [12]^7 &\equiv [0], \\
            [16]^7 &\equiv [0], \\
            [20]^7 &\equiv [0], \\
            [24]^7 &\equiv [0].
        \end{align*}

        The elements of order \(7\) in \(\mathbb{Z}/28\mathbb{Z}\) are \([4]\) and \([24]\).
    \end{proof}

    \item How many subgroups are there of order 7 in \(\mathbb{Z}/28\mathbb{Z}\)?
    \begin{proof}
        The number of subgroups of order \(7\) in \(\mathbb{Z}/28\mathbb{Z}\) corresponds to the number of elements of order \(7\). Since \(7\) is prime and divides \(28\), there are \( \phi(7) = 6\) distinct elements of order \(7\).

        Hence, there are \(6\) subgroups of order \(7\) in \(\mathbb{Z}/28\mathbb{Z}\).
    \end{proof}
    
\end{enumerate}
z/3 x z/5z 
\section*{Exercise 32}
\begin{enumerate}
    \item prove that the cyclic group $(\mathbb{Z} / 15\mathbb{Z})^*$ is isomorphic to the product group $\mathbb{Z} / 3\mathbb{Z} \times \mathbb{Z} / 5\mathbb{Z}$.
    \begin{proof}
        \leavevmode \\ 
        We want to show that the cyclic group $(\mathbb{Z} / 15\mathbb{Z})^*$ is isomorphic to the product group $\mathbb{Z} / 3\mathbb{Z} \times \mathbb{Z} / 5\mathbb{Z}$. We know that $(\mathbb{Z} / 15\mathbb{Z})^*$ is the set of all elements in $\mathbb{Z} / 15\mathbb{Z}$ that are relatively prime to $15$. We can find a generator for $(\mathbb{Z} / 15\mathbb{Z})^*$ by finding an element of order $8$. We can find an element of order $8$ by checking the orders of the elements in $(\mathbb{Z} / 15\mathbb{Z})^*$:
        \begin{align*}
            [1]^1 &= [1] \\
            [2]^4 &= [1] \\
            [4]^2 &= [1] \\
            [7]^4 &= [1] \\
            [8]^2 &= [1] \\
            [11]^4 &= [1] \\
            [13]^4 &= [1] \\
            [14]^2 &= [1]
        \end{align*}
        We can see that $[2]$ is an element of order $8$ in $(\mathbb{Z} / 15\mathbb{Z})^*$. Therefore, $(\mathbb{Z} / 15\mathbb{Z})^*$ is a cyclic group with generator $[2]$. We can now define a group isomorphism $\phi: (\mathbb{Z} / 15\mathbb{Z})^* \to \mathbb{Z} / 3\mathbb{Z} \times \mathbb{Z} / 5\mathbb{Z}$ by:
        \begin{align*}
            \phi([2]^0) &= ([0], [0]) \\
            \phi([2]^1) &= ([1], [2]) \\
            \phi([2]^2) &= ([2], [4]) \\
            \phi([2]^3) &= ([0], [1]) \\
            \phi([2]^4) &= ([1], [3]) \\
            \phi([2]^5) &= ([2], [1]) \\
            \phi([2]^6) &= ([0], [2]) \\
            \phi([2]^7) &= ([1], [4])
        \end{align*}
        We can see that $\phi$ is a group isomorphism. Therefore, $(\mathbb{Z} / 15\mathbb{Z})^*$ is isomorphic to the product group $\mathbb{Z} / 3\mathbb{Z} \times \mathbb{Z} / 5\mathbb{Z}$.

    \end{proof}
    \item Prove that the group $(\mathbb{Z} / 15\mathbb{Z})^*$ is isomorphic to the product group $\mathbb{Z} / 2 \mathbb{Z} \times \mathbb{Z} / 4\mathbb{Z}$. Conclude that $(\mathbb{Z} / 15\mathbb{Z})^*$ is not cyclic 
    \begin{proof}
        \leavevmode \\ 
        We want to show that the group \((\mathbb{Z}/15\mathbb{Z})^*\) is isomorphic to \(\mathbb{Z}/2\mathbb{Z} \times \mathbb{Z}/4\mathbb{Z}\).

From Part 1, we know \((\mathbb{Z}/15\mathbb{Z})^*\) has \(8\) elements and can be expressed as:
\[
(\mathbb{Z}/15\mathbb{Z})^* \cong \mathbb{Z}/3\mathbb{Z} \times \mathbb{Z}/5\mathbb{Z}.
\]
We will verify that this is also isomorphic to \(\mathbb{Z}/2\mathbb{Z} \times \mathbb{Z}/4\mathbb{Z}\) through structure analysis.

The orders of elements in \((\mathbb{Z}/15\mathbb{Z})^*\) reveal that there are \(2\) elements of order \(2\) and \(4\) elements of order \(4\). Hence, the group cannot be cyclic since it contains non-cyclic subgroups.

The structure \(\mathbb{Z}/2\mathbb{Z} \times \mathbb{Z}/4\mathbb{Z}\) is not cyclic either, hence concluding:
\[
(\mathbb{Z}/15\mathbb{Z})^* \cong \mathbb{Z}/2\mathbb{Z} \times \mathbb{Z}/4\mathbb{Z} \text{ and is not cyclic.}
\]
    \end{proof}
\end{enumerate}



\end{document}
