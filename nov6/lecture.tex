\documentclass{article}
\usepackage{amsmath, amsthm, amssymb, amsfonts}
\usepackage{thmtools}
\usepackage{graphicx}
\usepackage{setspace}
\usepackage{geometry}
\usepackage{float}
\usepackage{hyperref}
\usepackage[utf8]{inputenc}
\usepackage[english]{babel}
\usepackage{framed}
\usepackage[dvipsnames]{xcolor}
\usepackage{tcolorbox}
\usepackage{amsmath}
\usepackage{array}
\usepackage{tikz} 
\usepackage{multirow}
\usepackage{tcolorbox}
\usepackage{xcolor}
\usepackage{tikz-cd}

% Define a new color for the example box



\colorlet{LightGray}{White!90!Periwinkle}
\colorlet{LightOrange}{Orange!15}
\colorlet{LightGreen}{Green!15}

\newcommand{\HRule}[1]{\rule{\linewidth}{#1}}

\colorlet{LightGray}{black!10}
\colorlet{LightOrange}{orange!15}
\colorlet{LightGreen}{green!15}
\colorlet{LightBlue}{blue!15}
\colorlet{LightCyan}{cyan!15}



\declaretheoremstyle[name=Theorem,]{thmsty}
\declaretheorem[style=thmsty,numberwithin=section]{theorem}
\usepackage{tcolorbox} % Add missing package
\tcolorboxenvironment{theorem}{colback=LightGray}

\declaretheoremstyle[name=Definition,]{thmsty}
\declaretheorem[style=thmsty,numberwithin=section]{definition}
\tcolorboxenvironment{definition}{colback=LightBlue}

\declaretheoremstyle[name=Proposition,]{prosty}
\declaretheorem[style=prosty,numberlike=theorem]{proposition}
\tcolorboxenvironment{proposition}{colback=LightOrange}


\declaretheoremstyle[name=Example,]{prosty}
\declaretheorem[style=prosty,numberlike=theorem]{example}
\tcolorboxenvironment{example}{colback=LightOrange}

\declaretheoremstyle[name=Axiom,]{prcpsty}
\declaretheorem[style=prcpsty,numberlike=theorem]{axiom}
\tcolorboxenvironment{axiom}{colback=LightGreen}

\declaretheoremstyle[name=Lemma,]{prcpsty}
\declaretheorem[style=prcpsty,numberlike=theorem]{lemma}
\tcolorboxenvironment{lemma}{colback=LightCyan}





\setstretch{1.2}
\geometry{
    textheight=9in,
    textwidth=5.5in,
    top=1in,
    headheight=12pt,
    headsep=25pt,
    footskip=30pt
}

% ------------------------------------------------------------------------------

\begin{document}

% ------------------------------------------------------------------------------
% Cover Page and ToC
% ------------------------------------------------------------------------------

\title{ \normalsize \textsc{}
		\\ [2.0cm]
		\HRule{1.5pt} \\
		\LARGE \textbf{\uppercase{RINGS}}
		\HRule{2.0pt} \\ [0.6cm] \LARGE{Cosets}
		}

\date{\today}
\author{\textbf{Author} \\ 
		Tom Jeong
        }

\maketitle
\newpage

\tableofcontents
\newpage

% ------------------------------------------------------------------------------
\section{prime elemnets unique factorization domains.}

\underline{exercise}
show that the field of fractions of $\mathbb{Z}(i)$ is $\mathbb{Q}[i]$ 

\begin{definition}
    \begin{enumerate}
        \item A non-zero element in $x \in R - R^*$ Has a factorization into irreducible element if $\exists $ irreducible elements $$p_1, p_2, \dots, p_n \in R$$ such that $x = p_1p_2 \dots p_n$
        \item We say x has a \underline{unique } factorization into irreducible elements if for any other irreducieble factorization $$x = q_1q_2 \dots q_n$$  Every $p_i, i = 1, \dots n$ divides some $q_j$ for some $j = 1, \dots, m$
        \item[remark] Since $q_i$ is irreducible, $q_i \mid q_j$ implies $q_j = u p_i$ for some $u \in R^*$ therefore n = m
    \end{enumerate}


\end{definition}

\begin{definition}
    A domain $R$ such that every nonzero element in $R - R^*$ has a unique facrotization into irreducible elements is called a \underline{unique factorization domain UFD} 


\end{definition}
\begin{definition}
    A Nonzero element $p \in R- R^*$ is called a \underline{prime element} if $p \mid xy \rightarrow p \mid x \vee  p \mid y$ 
    
\end{definition}
\underline{exercise} If p is prime and $p \mid x_1x_2 \dots x_n, $ then $p \mid x_1$ for some $i = 1 \dots ,n$ 





\begin{proposition}(3.5.2) \leavevmode \\ 
    A prime element is irreducible
    h
    
\end{proposition}
\begin{proof}
    let $p \in R - R^*$ be prime \\ If $p \mid xy$, then $p \mid x$ or $p \mid y$ \\  wlog $p \mid x$
    \\ 
    Then $x = rp$ for some $r \in R$ \\ 
    Then $ p = rpy$ AND since R is a domain we can cancel. thus $1 = ry$ \\ 
    Hence, $y$ is a unit so p is irreducible 
\end{proof}
\underline{example} 
\begin{align*}
    &\mathbb{Z}[\sqrt{-5}] = \{a + b\sqrt{-5} | a, b \in \mathbb{Z}\} \subset \mathbb{C} \\ 
    & 6 = (1 - \sqrt{-5})(1 + \sqrt{5}) \\ 
    &\text{Claim: } 2 \in \mathbb{Z}[\sqrt{-5}] \text{ is irreducible but no prime}
\end{align*}
\begin{proof}
    1. 2 is not prime: \\ 
    $2 \mid (1 - \sqrt{-5})(1 + \sqrt{5})$ but 2 does \underline{not} divide $1 - \sqrt{-5}$ nor $1 + \sqrt{-5}$ \\ 
    since $\frac{1}{2} \pm \frac{\sqrt{-5}}{2} \not \in \mathbb{Z}[\sqrt{-5}]$ \\
    2. 2 is irreducible \begin{align*}
        &N : \mathbb{Z}[\sqrt{-5}] \rightarrow \mathbb{N } \\ 
        &N(z) = z\bar{z} \\ 
        &N (a + b\sqrt{-5}) = a^2 + 5b^2
    \end{align*}
    $z \in \mathbb{Z}[\sqrt{-5}]^* \leftrightarrows N(z) = 1$ \\ 
    Then $a + b \sqrt{-5}$ is a unit $\leftrightarrows a = \pm 1, b = 0$ \\ 
    Let \begin{align*}
        2 & = xy \\ 
        & \text{ where }  x = a + b \sqrt{-5} \\ 
        &y = c + d \sqrt{-5}\\ 
        \text{ Then } N(2) &= 4 = N(x) N(y) \\ 
        & = (a^2 + 5b^2) (c^2 + 5d^2) \\ 
        & = a^2c^2 + 5a^2d^2 + 5b^2c^2 + 25 b^2 d^2 \\ 
        &\text{therefore, } \\ 
        & 4 =a^2c^2 \text{ and } b = d= 0 \\ 
        & ac = \pm 2 \text{ so a or c is } \pm 1
    \end{align*}
    So x, or y is a unit
\end{proof}

\section{principle ideal domains and UFD}
Goal: we will show that every PID is a UFD  (the converse is not true ) \\ 
\underline{example } $\mathbb{R}[x,y] = \{\text{polynomials in x and y with real coefficients}\}$ is a UFD but is \underline{not } a PID
$$I = <x,y>$$ is not principal

\begin{lemma}[3.5.5]

    Let $R$ be a PID and $r \in R$ a non zero element, Then r has an irreducible factorization. 

\end{lemma}
    
        Claim: IF $<a_1> \subseteq <a_2> \subseteq \dots \in R$ where R is a PID, Then $$\exists N \in \mathbb{N}$$ such that $<a_i> = <a_{i+1}> \forall i > N$ 
    \begin{proof}[proof of claim] \leavevmode \\ 
        in the HW we shouwed that the $$\Cup_{i=1}^{\infty} <a_i> $$ is an ideal  \\ since R is a PID,$$\Cup_{i=1}^{\infty} <a_i>  = <d> $$ for some $d \in R$ \\ 
        Thus $d \in <a_N>$ for some $N$ , and so $<d> \subseteq <a_N>$ and hence $<a_i> = <d>$ for $i \geq N$  \\ 
        Suppsoe $r \in R - R^*$ is a non zero element which is not a product of irreducible.  then \\ 
        $$r = a_1b_1, a_1, b_1 \not \in R^*$$ 
        where at least one of $a_1, b_1$ is not a product of irreducibles. \\ 
        WLOG $a_1$ is not a product of irreducibles. \\ Then $a_1 = a_2b_2, a_2,b_2 \not \in R^*$  where at least of $a_2,b_2$ is not a product of irreducibles .. 
        \\ 
        Then 
        $$<R> \subsetneq <a_1> \subsetneq <a_2> \subsetneq <a_3> \subsetneq \dots $$ which contradicts the claim hence rm ust have an irreducible factorization.

    \end{proof}
    \begin{proposition}[3.5.6] \leavevmode\\ 
        suppose R is a PID that is not a field. An ideal $<x> \subset R$ is a Maximal Ideal IFF x is irreducible. 

    \end{proposition}
    \begin{proof}
        \begin{enumerate}
            \item $ \rightarrow$ Assume $<x>$ is a maximal and $x = ab$, we want to show that a or b is a unit. \begin{align*}
                &\text{ Assume neither } a \text{ or } b \text{ is a unit } \\ 
                &\text{Then } <x> \subsetneq <a> \text{ since } b \text{ is not a unit} \\ 
                &<x> \subsetneq <b> \text{ since } a \text{ is not a unit}
            \end{align*}
            contradicting maximaility of $<x>$ hence a or b must be a unit 
            \item $\leftarrow$ Suppose $x$ is irreducible. \begin{align*}
                &\text{IF } <x> \subseteq <y> \text{ then } \\ 
                &x = \lambda y  \text { for some } \lambda \in R \\ 
                &\text{since } x \text{ is irreducible } \\ 
                &\lambda \text{ or } y \text{ is a unit} \\ 
                &\text{If } \lambda \text{ is a unit, then } <x> = <y> \\ 
                &\text{If } y \text{ is a unit, } <y> = K \\ 
                &\text{Hence  } <x> \text{ is maximal. }
            \end{align*}
        \end{enumerate}


    \end{proof}

    \begin{theorem}[3.5.7]
        A PID R is a UFD 
        
    \end{theorem}
    \begin{proof}
        By lemma 3.5.5 since R is  PID, irreducible factorization exists. We need to show uniqueness. We will show that irreducibility elemens are prime and then apply Prop 3.5.3 to show that R is a UFD 
        
    \end{proof}

    \begin{proposition}[3.5.3]
        Let $R$ be a ring where every non-zero element $r \in R - R^*$ has a factorization into irreducibles. Every irreducible element is a prime element in R iff R is a UFD
        
    \end{proposition}
    \begin{proof}
        \begin{enumerate}
            \item $\rightarrow$ Suppose $x \in R - R^*$ with $x =p_1p_2 \dots p_n = q_1q_2 \dots q_m$ wher pi qj are irreducible. 
            \item $\leftarrow $ Let R be a ufd and $p \in R$ irreducible. Suppose $p \mid xy$ since R is a UFD, x and y each have unieque factorizations and by uniqueness one of these factorizations must have an irreducible factor that is divisible by p. \\ Hence $p \mid x$ or $p \mid y$ so p is prime.            
        \end{enumerate}
    \end{proof}

    
    To see that every PID is a UFD we will show that irreducible elements are prime and apply prop 3.5.3 to show R is a UFD. \\ 
    Let $p \in R$ p irreducible, with $p \mid ab $ and $p \nmid a$ \\ WTS 
    $p\mid b$  \\ since $p \nmid a, a \not \in <p> $  \\ 
    Then, $<p> \subsetneq <a, p>$ \\ Since $p$ is irreducible, $<p> $ is maximal by te earlier prop 3.5.6 \\ Hence $<a, p> = R$ so exists $x,y \in R $ so that xa + yp = 1 

    
    Multiplying both sides by b: $xab + ypb = b$  \\ 
    Since $p \mid ab$, it follows that $p \mid b$   
\end{document}






 