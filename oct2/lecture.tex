\documentclass{article}
\usepackage{amsmath, amsthm, amssymb, amsfonts}
\usepackage{thmtools}
\usepackage{graphicx}
\usepackage{setspace}
\usepackage{geometry}
\usepackage{float}
\usepackage{hyperref}
\usepackage[utf8]{inputenc}
\usepackage[english]{babel}
\usepackage{framed}
\usepackage[dvipsnames]{xcolor}
\usepackage{tcolorbox}
\usepackage{amsmath}
\usepackage{array}
\usepackage{multirow}


\colorlet{LightGray}{White!90!Periwinkle}
\colorlet{LightOrange}{Orange!15}
\colorlet{LightGreen}{Green!15}

\newcommand{\HRule}[1]{\rule{\linewidth}{#1}}

\colorlet{LightGray}{black!10}
\colorlet{LightOrange}{orange!15}
\colorlet{LightGreen}{green!15}
\colorlet{LightBlue}{blue!15}
\colorlet{LightCyan}{cyan!15}



\declaretheoremstyle[name=Theorem,]{thmsty}
\declaretheorem[style=thmsty,numberwithin=section]{theorem}
\usepackage{tcolorbox} % Add missing package
\tcolorboxenvironment{theorem}{colback=LightGray}

\declaretheoremstyle[name=Definition,]{thmsty}
\declaretheorem[style=thmsty,numberwithin=section]{definition}
\tcolorboxenvironment{definition}{colback=LightBlue}

\declaretheoremstyle[name=Proposition,]{prosty}
\declaretheorem[style=prosty,numberlike=theorem]{proposition}
\tcolorboxenvironment{proposition}{colback=LightOrange}


\declaretheoremstyle[name=Proof,]{prosty}
\declaretheorem[style=prosty,numberlike=theorem]{proofbox}
\tcolorboxenvironment{proofbox}{colback=LightOrange}

\declaretheoremstyle[name=Axiom,]{prcpsty}
\declaretheorem[style=prcpsty,numberlike=theorem]{axiom}
\tcolorboxenvironment{axiom}{colback=LightGreen}

\declaretheoremstyle[name=Lemma,]{prcpsty}
\declaretheorem[style=prcpsty,numberlike=theorem]{lemma}
\tcolorboxenvironment{lemma}{colback=LightCyan}





\setstretch{1.2}
\geometry{
    textheight=9in,
    textwidth=5.5in,
    top=1in,
    headheight=12pt,
    headsep=25pt,
    footskip=30pt
}

% ------------------------------------------------------------------------------

\begin{document}

% ------------------------------------------------------------------------------
% Cover Page and ToC
% ------------------------------------------------------------------------------

\title{ \normalsize \textsc{}
		\\ [2.0cm]
		\HRule{1.5pt} \\
		\LARGE \textbf{\uppercase{Symmetric Groups continued..}}
		\HRule{2.0pt} \\ [0.6cm] \LARGE{Cosets}
		}

\date{\today}
\author{\textbf{Author} \\ 
		Tom Jeong
        }

\maketitle
\newpage

\tableofcontents
\newpage

% ------------------------------------------------------------------------------
\section{symmetric group}
\begin{lemma}[2.9.8] \leavevmode \\
    suppose $\tau = (i, i_2, \dots, i_k)$ is a k-cycle and $\sigma \in S_n$ then, $\sigma \tau \sigma^{-1} = (\sigma(i), \sigma(i_2), \dots, \sigma(i_k))$.
    where $\sigma(i_j) \rightarrow \sigma(i_{j+1})$
\end{lemma}
ex. $$(1 3 4) = (1 \space 2) (2 \space 3\space 4) (1\space2)$$
\begin{proof}
    let $j = \{\sigma(i_1), \sigma(i_2), \dots, \sigma(i_k)\}$, both the LHS and RHS of the lemma formula take on the same values for $j\in J$ and $j \notin J$ to itself. Hence LSH are RHS are equal. 
\end{proof}
Another way to see that every permutation is a product of simple transpositions: (idea of bubble sort)
ex: $$\sigma = \begin{bmatrix}
    1 & 2 & 3 & 4 & 5 \\
    5 & 2 & 1 & 4 & 3
\end{bmatrix}$$ 
use bubble sort: each time a neighboring pair is out of order, swap the two, and iterate the process. 
\begin{align*}
    52143 &\rightarrow^{(12)} 25143 \\ 
    \sigma = &\begin{bmatrix}
        1 & 2 & 3 & 4 & 5 \\
        5 & 2 & 1 & 4 & 3
    \end{bmatrix} (12) = \sigma = \begin{bmatrix}
        1 & 2 & 3 & 4 & 5 \\
        2 & 5 & 1 & 4 & 3
    \end{bmatrix} \\ 
    25143 &\rightarrow^{(23)} 21543  \rightarrow^ {(34)} 21354 \rightarrow^{(45)} 21345 \dots   \\ 
    \begin{bmatrix}
        1 & 2 & 3 & 4 & 5 \\
        5 & 2 & 1 & 4 & 3
    \end{bmatrix}&(12)(23)(12)(34)(45)(34) = e 
\end{align*}
so equivelently $\begin{bmatrix}
    1 & 2 & 3 & 4 & 5 \\
    5 & 2 & 1 & 4 & 3
\end{bmatrix} = (12)(23)(12)(34)(45)(34)$ \\



Q: what is the smallest number of simple transpositions needed to express $\sigma \in S_n$?
\begin{definition}[2.9.10] \leavevmode \\     
    let $\sigma \in S_n$ A pair of indices $(i,j)$ where $1 \leq i < j \leq n$ is called \underline{inversion of $\sigma$} if $\sigma(i) > \sigma(j)$. \\ 
    Let $I_{\sigma} = \{(i,j) | 1 \leq i < j \leq n \text{ and } \sigma(i) > \sigma(j) \}$ denote the set of inversions of $\sigma$ and $n_{\sigma} = |I_{\sigma}|$ denote the number of inversions of $\sigma$.
\end{definition}
ex: $\sigma = \begin{bmatrix}
    1 & 2 & 3 \\ 
    5 & 2 & 1 \\
\end{bmatrix} $ then we have $I_\sigma = \{(12), (13), (23)\}$ 
\begin{proposition}[2.9.12] \leavevmode 
    \begin{enumerate}
        \item the permutation $\sigma \in S_n$ is the identity iff $n_{\sigma} = 0$
        \item if $\sigma \not = id$ then $\exists i \in \{1, 2 \dots, n-1\}$ such that $\sigma(i) > \sigma(i+1)$
    \end{enumerate}
    
\end{proposition}

\begin{proof}
    \begin{enumerate}
        \item $\rightarrow$: true since identity has no inversions 
        \\ 
    $\leftarrow$ if $\sigma \not = id$ then $\exists$ smallest $i \in M_n$ such that $\sigma(i) > j$ then $(i, \sigma^{-1} (i))$ is an inversion of $\sigma$, hence $n_{\sigma} \not = 0$
    \item if $\sigma(1) < \sigma(2) < \dots < \sigma(n)$ then $n_{\sigma} = 0$ so $\sigma = id$
    \end{enumerate}
\end{proof}

\begin{lemma}[2.9.13] \leavevmode \\ 
    Let $s_i \in S_n$ be the imple transposition $(i, i+1)$ then $n_{\sigma s_i} = \begin{cases}
        n_{\sigma} + 1 & \text{if } \sigma(i) < \sigma(i+1) \\ 
        n_{\sigma} - 1 & \text{if } \sigma(i) > \sigma(i+1) \\
    \end{cases}$
\end{lemma}
\begin{proof}
    suppose $\sigma(i) < \sigma(i + 1)$ then $(i, i+1)$ is not an inversion of $\sigma$ but is an inversion of $\sigma s_i$ so $n_{\sigma s_i} = n_{\sigma} + 1$ \\
\end{proof}
caimL 
$$\phi: I_{\sigma} \rightarrow I_{\sigma s_i} \backslash \{(i, i+1)\}$$
$$(k,l) \rightarrow (s_i(k) , s_i(l) ) $$ is a bijection and so $$n_{\sigma s_i} = n_{\sigma} + 1$$



\begin{proof}
    proof of claim: 
    If $(kl) \in I_{\sigma}$ then $s_i(k) >  s_i(l)$ since theo nly way for $s_i(k) < s_i(l)$ is if $k = i$ and $l = i+1$ but $(kl) \not = (i, i+1)$
    Hence $(s_i(k), s_i(l)) \in I_{\sigma s_i} \backslash \{(i, i+1)\}$. Similarly if $(s_i(k), s_i(l)) \in I_{\sigma s_i} \backslash \{(i, i+1)\}$ then $s_i(k) > s_i(l)$ so $(kl) \in I_{\sigma}$
\end{proof}
\begin{proposition}[2.9.14] \leavevmode \\ 
    let $\sigma \in S_n$ THen \begin{enumerate}
        \item $\sigma$ is a product of $n_{\sigma}$ simple transpositions
        \item $n_{\sigma}$ is the minimal number of simple transpositions needed to write $\sigma$ as a product of simple transpositions. 
        
    \end{enumerate}
    
\end{proposition}
\begin{proof}
    1. Induction on $n_{\sigma}$ \\
    Base Case: if $n_{\sigma} = 0, $ then by proposition 2.9.12 $\sigma = id$ so $\sigma $ is the empty product of zero simple transpositions. 
    \\ 
    Inductive step: if $n_{\sigma} != 0$ then by prop 2.9.12 there exists $i \in \{1, 2, \dots, n-1\}$ such that $\sigma(i) > \sigma(i+1)$ then by lemma 2.9.13 $n_{\sigma s_i} = n_{\sigma} - 1$ so by induction $\sigma = s_i \sigma s_i$ is a product of $n_{\sigma} - 1$ simple transpositions.

    By the induction hypothesis $\sigma s_i$ can be written as a product of $n_{\sigma s_i} = n_{\sigma} - 1$ simple transpositions, then $\sigma s_i s_i = \sigma $ is a product of $n_{\sigma}$ simple transpositions.


    2. let $l(\sigma) = $ the min number of simple transpositions needed to express $\sigma$. By part1, $l(\sigma) \leq n_{\sigma}$. Suppose $l(\sigma) = k < n_{\sigma}$ then $\sigma = s_{i_1} s_{i_2} \dots s_{i_k}$ is a product


    inductive step: assume $l(\sigma) > 0.$ then we can find a simple transposition $s_i$ such that $l(\sigma s_i)  = l(\sigma) - 1$. Therefore, $l(\sigma s_i) = n_{\sigma s_i}$ by induction hypothesis and $l(\sigma) \geq n_{\sigma}$ by lemme 2.9. 13
    $$l(\sigma ) - 1 = l(\sigma s_i) = n \sigma s_i = n_{\sigma} \pm 1$$
\end{proof}


\begin{definition}[2.9.15]
    \leavevmode \\ 
    the signs of $\sigma \in S_n$ is $sgn(\sigma) = (-1)^{n_{\sigma}}$
\end{definition}


\begin{proposition}[2.9.16 ] \leavevmode \\ 
    the sign of a permutation $$sgn: S_n \rightarrow \{\pm 1\}$$ is a group homomorphism where composition on $\{\pm 1\}$ is multiplication s
\end{proposition}

\begin{proof}
    by lemme 2.9.13, $n_{\sigma s_i} = n_{\sigma} \pm1 $ so $sgn(\sigma s_i) = (-1)^{n_{\sigma} \pm 1} = (-1)^{n_{\sigma}} (-1)^{\pm 1} = sgn(\sigma) sgn(s_i)$
    since any $\tau \in S_n$ can be written as a product of simple transpositions it follows that $sgn(\tau) = sgn(\sigma_1) sgn(\sigma_2) \dots sgn(\sigma_k)$
\end{proof}

\begin{definition}
    the alternating group is $$A_n = ker(sgn) = \{\sigma \in S_n | sgn(\sigma ) = 1\}$$
\end{definition}

\underline{Note:} $A_n $ is a normal subgroup of $S_n$ by the isomorphism theorem, $S_n / A_n \cong \{\pm 1\}$
 \end{document}
 