\documentclass{article}
\usepackage{amsmath, amsthm, amssymb, amsfonts}
\usepackage{thmtools}
\usepackage{graphicx}
\usepackage{setspace}
\usepackage{geometry}
\usepackage{float}
\usepackage{hyperref}
\usepackage[utf8]{inputenc}
\usepackage[english]{babel}
\usepackage{framed}
\usepackage[dvipsnames]{xcolor}
\usepackage{tcolorbox}
\usepackage{amsmath}
\usepackage{array}
\usepackage{tikz} 
\usepackage{multirow}
\usepackage{tcolorbox}
\usepackage{xcolor}
\usepackage{tikz-cd}

% Define a new color for the example box



\colorlet{LightGray}{White!90!Periwinkle}
\colorlet{LightOrange}{Orange!15}
\colorlet{LightGreen}{Green!15}

\newcommand{\HRule}[1]{\rule{\linewidth}{#1}}

\colorlet{LightGray}{black!10}
\colorlet{LightOrange}{orange!15}
\colorlet{LightGreen}{green!15}
\colorlet{LightBlue}{blue!15}
\colorlet{LightCyan}{cyan!15}



\declaretheoremstyle[name=Theorem,]{thmsty}
\declaretheorem[style=thmsty,numberwithin=section]{theorem}
\usepackage{tcolorbox} % Add missing package
\tcolorboxenvironment{theorem}{colback=LightGray}

\declaretheoremstyle[name=Definition,]{thmsty}
\declaretheorem[style=thmsty,numberwithin=section]{definition}
\tcolorboxenvironment{definition}{colback=LightBlue}

\declaretheoremstyle[name=Proposition,]{prosty}
\declaretheorem[style=prosty,numberlike=theorem]{proposition}
\tcolorboxenvironment{proposition}{colback=LightOrange}


\declaretheoremstyle[name=Example,]{prosty}
\declaretheorem[style=prosty,numberlike=theorem]{example}
\tcolorboxenvironment{example}{colback=LightOrange}

\declaretheoremstyle[name=Axiom,]{prcpsty}
\declaretheorem[style=prcpsty,numberlike=theorem]{axiom}
\tcolorboxenvironment{axiom}{colback=LightGreen}

\declaretheoremstyle[name=Lemma,]{prcpsty}
\declaretheorem[style=prcpsty,numberlike=theorem]{lemma}
\tcolorboxenvironment{lemma}{colback=LightCyan}





\setstretch{1.2}
\geometry{
    textheight=9in,
    textwidth=5.5in,
    top=1in,
    headheight=12pt,
    headsep=25pt,
    footskip=30pt
}

% ------------------------------------------------------------------------------

\begin{document}

% ------------------------------------------------------------------------------
% Cover Page and ToC
% ------------------------------------------------------------------------------

\title{ \normalsize \textsc{}
		\\ [2.0cm]
		\HRule{1.5pt} \\
		\LARGE \textbf{\uppercase{RINGS}}
		\HRule{2.0pt} \\ [0.6cm] \LARGE{Cosets}
		}

\date{\today}
\author{\textbf{Author} \\ 
		Tom Jeong
        }

\maketitle
\newpage

\tableofcontents
\newpage

% ------------------------------------------------------------------------------
\section{RINGS}
Q: Given a ring R, is there a field F such that $R \subseteq F$ ?\\
ex. $\mathbb{Z} \subseteq \mathbb{Q}$ \\ 
\underline{the field of Fractions } \\ 
Q: How do we build $\mathbb{Q}$ from $\mathbb{Z}$
$$\frac{p}{q} \in \mathbb{Q}, p \in \mathbb{Z}, q \in \mathbb{Z} \backslash \{0\} $$
$$\mathbb{Q} = \mathbb{Z} \times (\mathbb{Z} - \{0\}) / \sim$$
$(p, q) \sim (r,s)$ if $ps =qr$
More generally, for a domain $R$ consider \begin{align*}
    & R \times (R - \{0\}) \\ 
    &  \text{ and define an equivalence relation} \\ 
    &(a, s) \sim (b,t) \text{ iff } at = sb \\ 
\end{align*}
\underline{exercise } Check that $\sim$ is an equivalence relation. 
\\ 
define the \underline{field of fractions } Q of a domain R as $Q = R \times ( R - \{0\} ) / \sim $ \\ 
define $(a,s) = \frac{a}{s}$
\\ 
\underline{Exercise}: Q is a ring with operations: 
\begin{align*}
    &\frac{a}{s} + \frac{b}{t} = \cfrac{at + bs}{st} \text{ or } (a,s) + (b,t) = (at + bs, st)\\
    &\frac{a}{s} \cdot \frac{b}{t} = \cfrac{ab}{st}
\end{align*}


\begin{enumerate}
    \item additive identity in Q is $\cfrac{0}{1} = \cfrac{0}{s} \text{ , } \forall s \in R - \{0\}$ 
    \item multiplication in Q is $1 = \cfrac{1}{1} = \cfrac{s}{s}  \text{ , } \forall s \in R - \{0\}$
\end{enumerate}

\begin{lemma} \leavevmode \\ 
    Q is a field
\end{lemma}
\begin{proof} \leavevmode \\
    If $\cfrac{a}{s} \not = 0$, then $a \not= 0$ \\ 
    and then $\cfrac{s}{a} \cdot \cfrac{a}{s} = 1$
\end{proof}



\underline{Exercise}
How to define this injective ring homomorphisms \\ 
$i: R \to Q$ , and $i(a) = \cfrac{a}{1}$ 

\begin{proposition}[3.4.1]\leavevmode \\ 
    Let $R$ be a domain with field of fractions $Q$. Let $L$ be a field and let $$\varphi: R \to L$$ be an injectivre ring homomorphisms, 
    then, $\exists!$ injective ring homomorphism $$\bar{\varphi}: Q \to L $$ such that $\bar{\varphi} \cdot i = \varphi$ 
    \begin{tikzcd}
        R \arrow[r, "i"] \arrow[rd, "\varphi"'] & Q \arrow[d, "\exists! \bar{\varphi}"] \\
        & L
    \end{tikzcd}

   
\end{proposition}

    \underline{EX} 
    \begin{align*}
        &R = \mathbb{Z} \\ 
        &Q = \mathbb{Q} \\ 
        &L = \mathbb{R} 
    \end{align*}


    \begin{proposition}[3.4.2 COROLLARY not prop] \leavevmode \\ 
        Let $R$ be a domain contained in a field $L$. The smallest subfield in $L$ containing $R$ is \\ 
        $$K = \{as^{-1} | a \in R, s \in R - \{0\}\} $$
        The field of fractino is isomorphic to K. 
    \end{proposition}

    \begin{proof}
        \leavevmode \\ 
        Let $a, b \in R$ and $s, t \in R -\{0\}$ \\ 
        Then $(as^{-1})(bt^{-1}) = abs^{-1}t^{-1}$ \\ 
        And $as^{-1} + bt^{-1} = (at + bs) (st)^{-1}$ \\ 
        and $(as^{-1})^{-1} = sa^{-1} $ if $ a\not = 0$ \\ 
        Thus $K$ is a subfield of $L$ \\ 
        Note that any subfield of L containing R must contain K, and we have $a \in R, s^{-1} \forall s \in R - \{0\}$ \\ 


Let $Q$ be the field of fractions of R and ocntains the injection $R \to L$ sending $r \in R$ to $r \in L$ \\
    By Proposition (3.4.1) $\bar{\varphi }(\frac{a}{s}) = as^{-1}$ is an injective homomorphism.
and is clearly injective onto K. \\ 




    \end{proof}

\underline{Sket Proof of Proposition}
\\ 
since $\varphi \cdot i = \varphi$ we must have for $s \in R - \{0\}$, \begin{align*}
    1 &= \bar{\varphi } (1) \\ 
    & = \bar{\varphi}(\frac{s}{1} \cdot \frac{1}{s})  \\ 
    & = \bar{\varphi}(\frac{s}{1}) \cdot \bar{\varphi}(\frac{1}{s}) \\
    & = \bar{\varphi}(i(s)) \cdot \bar{\varphi}(\frac{1}{s}) \\ 
    & = \bar{\varphi}(s) \cdot \bar{\varphi}(\frac{1}{s})
    \end{align*}
Hence 
$$\bar{\varphi}(\frac{a}{s}) = \bar{\varphi}(\frac{a}{1})\bar{\varphi}(\frac{1}{s}) = \varphi(a) \varphi(s)^{-1}$$
We need to check $\bar{\varphi}$ is well-defined: \\
if $\frac{a}{s} = \frac{b}{t}$ then $at = sb$ , and so \\ 
$\varphi(a) \varphi(t) = \varphi(s) \varphi(b)$ and so \\ 
$\varphi(a) \varphi(s )^{-1} = \varphi(b) \varphi(t)^{-1} $

i.e. $\bar{\varphi(\frac{a}{s})} = \bar{\varphi(\frac{b}{t})}$

\underline{exercise}
Check that $\bar{\varphi} $ is a ring homomorphism and injective 


Plans for the rest of the semester: Given a ring R, if R is a field, every non-zero element has a multi. inverse \\ 
If R is a domain, we can build its field of fractions. 
\\ 
Q: What properites of $\mathbb{Z}$ Hold in other rings? 
\\ 
Let R be a ring \begin{enumerate}
    \item Can we factor elements in R? 
    \item Are factorizations unique? 
    \item Is there a notion of "prime" in R? 
    \item Does the division algorithm work in r do we have unique remainder
    
\end{enumerate}


Assume R is a domain for the next few classes
\underline{Factoring in R} 
Deifine: let $x, z \in R$ if $x = ry$ for some $r \in R$ we say y is a divisor of $x$  denoted $y \mid x$

 
Notes if R is a PID then for any a,b $\in R, \exists d$  such that $$<a, b> = \{\lambda_1 a + \lambda_2 b | \lambda_1, \lambda_2 \in R \}$$
\underline{exercise} If R is a PID then this d is exactly the GCD of a and b
\end{document}





 