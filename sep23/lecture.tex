\documentclass{article}
\usepackage{amsmath, amsthm, amssymb, amsfonts}
\usepackage{thmtools}
\usepackage{graphicx}
\usepackage{setspace}
\usepackage{geometry}
\usepackage{float}
\usepackage{hyperref}
\usepackage[utf8]{inputenc}
\usepackage[english]{babel}
\usepackage{framed}
\usepackage[dvipsnames]{xcolor}
\usepackage{tcolorbox}
\usepackage{amsmath}
\usepackage{array}
\usepackage{multirow}


\colorlet{LightGray}{White!90!Periwinkle}
\colorlet{LightOrange}{Orange!15}
\colorlet{LightGreen}{Green!15}

\newcommand{\HRule}[1]{\rule{\linewidth}{#1}}

\colorlet{LightGray}{black!10}
\colorlet{LightOrange}{orange!15}
\colorlet{LightGreen}{green!15}
\colorlet{LightBlue}{blue!15}
\colorlet{LightCyan}{cyan!15}



\declaretheoremstyle[name=Theorem,]{thmsty}
\declaretheorem[style=thmsty,numberwithin=section]{theorem}
\usepackage{tcolorbox} % Add missing package
\tcolorboxenvironment{theorem}{colback=LightGray}

\declaretheoremstyle[name=Definition,]{thmsty}
\declaretheorem[style=thmsty,numberwithin=section]{definition}
\tcolorboxenvironment{definition}{colback=LightBlue}

\declaretheoremstyle[name=Proposition,]{prosty}
\declaretheorem[style=prosty,numberlike=theorem]{proposition}
\tcolorboxenvironment{proposition}{colback=LightOrange}


\declaretheoremstyle[name=Proof,]{prosty}
\declaretheorem[style=prosty,numberlike=theorem]{proofbox}
\tcolorboxenvironment{proofbox}{colback=LightOrange}

\declaretheoremstyle[name=Axiom,]{prcpsty}
\declaretheorem[style=prcpsty,numberlike=theorem]{axiom}
\tcolorboxenvironment{axiom}{colback=LightGreen}

\declaretheoremstyle[name=Lemma,]{prcpsty}
\declaretheorem[style=prcpsty,numberlike=theorem]{lemma}
\tcolorboxenvironment{lemma}{colback=LightCyan}





\setstretch{1.2}
\geometry{
    textheight=9in,
    textwidth=5.5in,
    top=1in,
    headheight=12pt,
    headsep=25pt,
    footskip=30pt
}

% ------------------------------------------------------------------------------

\begin{document}

% ------------------------------------------------------------------------------
% Cover Page and ToC
% ------------------------------------------------------------------------------

\title{ \normalsize \textsc{}
		\\ [2.0cm]
		\HRule{1.5pt} \\
		\LARGE \textbf{\uppercase{Lecture 10 REVIEW sep 23}}
		\HRule{2.0pt} \\ [0.6cm] \LARGE{Cosets}
		}

\date{\today}
\author{\textbf{Author} \\ 
		Tom Jeong
        }

\maketitle
\newpage

\tableofcontents
\newpage

% ------------------------------------------------------------------------------
\section{Order of Element}
\begin{proposition}
    let $G$ be a cyclic group 
    \begin{enumerate}
        \item every subgroup of $G$ is cyclic
        \item suppose $G$ is finite and $d$ is a divisor of $|G|$, then there is a unique subgroup of $G$ of order $d$
        \item there are $\phi(d)$ elements of order d in $G$ These are exactly the generators of the unique subgroup of order $d$
        
    \end{enumerate}
\end{proposition}


Corollary 2.7.6 
lLet $N$ be a positive integer. Then $$\sum_{d|N}^{}\phi(d) = N$$ where the summ is over all divisors $d \in div(N)$
\begin{proof}
    \leavevmode\\
    let $G = \mathbb{Z} / N\mathbb{Z}$ 

    $N = \sum_{g \in G}^{} 1 = \sum_{d|N}^{} \underbrace{\sum_{g \in G}^{} 1}_{\phi(d)} = \sum_{d|N}^{}\phi(d) $
\end{proof}

\begin{theorem}[EUler 1.7.2]
    Let $a, n \in \mathbb{Z} $ be relatively prime and $n \in \mathbb{N}$ then $$a^{\phi(n) } \equiv 1 \text{ (mod n)}$$
    
\end{theorem}

\begin{proof} \leavevmode \\ 
    Let $G \mathbb{Z} / n\mathbb{Z} *$ Then $|G| = \phi(n)$  \\ 
    Since $gcd(a,n) = 1$ we have $[a] \in G$ \\ 
    By Proposition 2.6.3, $[1] = [a]^{|G|} = [a]^{\phi(n)}$ so $a^{\phi(n)} \equiv 1 \text{ (mod n)}$ 
    
\end{proof}
Take away: grousp can be a poweful tool for studying things that dont immediately seem to be about groups.
\begin{definition}
    \leavevmode \\ 
Let $G_1, G_2, \dots, G_n$ be a group; The \underline{Product} of $G_1, G_2, \dots, G_n$ is the group $G  = G_1 \times G_2 \times \dots \times G_n = \{(g_1, g_2, \dots, g_n) | g_i \in G_i \}$ with the composition law $(g_1, g_2, \dots, g_n) \cdot (h_1, h_2, \dots, h_n) = (g_1h_1, g_2h_2, \dots, g_nh_n)$


\end{definition}
\underline{exercise} the composition is associative. 
identity: $(e_1, e_2, \dots, e_n)$
inverse: $(g_1, g_2, \dots, g_n)^{-1} = (g_1^{-1}, g_2^{-1}, \dots, g_n^{-1})$ \\ 
ex: $\mathbb{Z} / 2\mathbb{Z} \times \mathbb{Z} / 2\mathbb{Z} = \{(0,0), (0,1), (1,0), (1,1)\}$  \\ 


$\mathbb{Z} / 2\mathbb{Z} \times \mathbb{Z} / 2\mathbb{Z}  \not \cong \mathbb{Z} / 4\mathbb{Z}$  since every element in the lhs has order 2. 
\underline{NOte: } if we have isomorphism $\phi_i: H \rightarrow G_i$ then we have an isomorphism $\phi: H \rightarrow G_1 \times G_2 \times \dots \times G_n$ given by $\phi(h) = (\phi_1(h), \phi_2(h), \dots, \phi_n(h))$ \\ 

\begin{lemma}[cor 1.5.11 ii ]\leavevmode \\ 
    If $a, b, c \in \mathbb{Z}$ and $gcd(a,b) = 1$ and $a | bc$ then $a | c$
 \end{lemma}
 \begin{proposition}[2.8.2]
    \leavevmode \\ 
    let $n_1, n_2, \dots, n_r \in \mathbb{Z} $ be pairwise relatively prime integers and let $N = n_1n_2 \dots n_r$ then for any $a_1, a_2, \dots, a_r \in \mathbb{Z}$ we have the system of congruences
    
 \end{proposition}
 if $\phi_i$ denotes the canonical hommomorphism $\phi_i: \mathbb{Z} \rightarrow \mathbb{Z} / n_i\mathbb{Z}$  with $\phi_i(x) = [x]$ then the map $$\tilde{\phi}:\mathbb{Z} / N\mathbb{Z} \rightarrow \mathbb{Z} / n_1 \mathbb{Z} \times \mathbb{Z} / n_2 \mathbb{Z} \times \dots \times \mathbb{Z} / n_r \mathbb{Z}$$ given by $\tilde{\phi}([x]) = ([x], [x], \dots, [x])$ is an isomorphism.

\begin{proof}
    claim Ker $\tilde{\phi} = N\mathbb{Z}$ 
\[
x \in N \mathbb{Z} \iff x \equiv 0 \text{ (mod } n_i\text{) for all } i \iff [x] = ([0], [0], \dots, [0]) \iff \tilde{\phi}([x]) = ([0], [0], \dots, [0]) 
\]
\[
\text{Thus, } \text{Ker } \tilde{\phi} = N \mathbb{Z} 
\]
\[
\text{By the Isomorphism Theorem, } \tilde{\phi} \text{ is an isomorphism.}
\]
\end{proof}
We'll state a big theorem about certain abelian groups .

\begin{definition}
Let $G$ be a group hand let $a_i \in G$ for $ i \in I$ The smallest subgroup of G containing $\{a_i | i \in I\} $ is the subgroup generated by $\{a_i | i \in I\}$ and is denoted $\{ a_i | i \in I \}$
If this subgroup is all of $G$ then we say that $G$ is generated by $\{a_i | i \in I\}$ 
If there is a finite set $\{a_i | i \in I\} $ that generates $G$ then we say that $G$ is finitely generated.
\
\end{definition}
\begin{theorem}
    If $G$ is a group and $a_i \in G$ for $i \in I$ then the subgorup $H$ generated by $\{a_i | i \in I \}$ has elements precisely these leements that are finite products of integral powers of $a_i$ wheere powers of a fixed $a_i$ may occur several times in the product. 


    
\end{theorem}
ex. $D_3$ is finitely gneeratedby $r_1, s_1$. ex. $\mathbb{Z}$ is generated by 1.
ex $\mathbb{Z} \times \mathbb{Z}$


\begin{theorem}[fundamental Theorem of Fintely Generated Abelian groups] \leavevmode \\ 
    every finitely generated abelian group is isomorphic to a direct product of cyclic groups in the form $$\mathbb{Z}^{n_1} \times \mathbb{Z} / n_2 \mathbb{Z} \times \dots \times \mathbb{Z} / n_r \mathbb{Z}$$ where $n_1, n_2, \dots, n_r \in \mathbb{N}$ and $n_1 | n_2 | \dots | n_r$
    Where the $p_i's$ are prime not necessarily distinct and $r_i$  are positive integers the direct product is unique up to recordings the factors. 
\end{theorem}
 \end{document}
 