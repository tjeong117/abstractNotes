\documentclass{article}
\usepackage{amsmath, amsthm, amssymb, amsfonts}
\usepackage{thmtools}
\usepackage{graphicx}
\usepackage{setspace}
\usepackage{geometry}
\usepackage{float}
\usepackage{hyperref}
\usepackage[utf8]{inputenc}
\usepackage[english]{babel}
\usepackage{framed}
\usepackage[dvipsnames]{xcolor}
\usepackage{tcolorbox}
\usepackage{amsmath}
\usepackage{array}
\usepackage{multirow}


\colorlet{LightGray}{White!90!Periwinkle}
\colorlet{LightOrange}{Orange!15}
\colorlet{LightGreen}{Green!15}

\newcommand{\HRule}[1]{\rule{\linewidth}{#1}}

\colorlet{LightGray}{black!10}
\colorlet{LightOrange}{orange!15}
\colorlet{LightGreen}{green!15}
\colorlet{LightBlue}{blue!15}
\colorlet{LightCyan}{cyan!15}



\declaretheoremstyle[name=Theorem,]{thmsty}
\declaretheorem[style=thmsty,numberwithin=section]{theorem}
\usepackage{tcolorbox} % Add missing package
\tcolorboxenvironment{theorem}{colback=LightGray}

\declaretheoremstyle[name=Definition,]{thmsty}
\declaretheorem[style=thmsty,numberwithin=section]{definition}
\tcolorboxenvironment{definition}{colback=LightBlue}

\declaretheoremstyle[name=Proposition,]{prosty}
\declaretheorem[style=prosty,numberlike=theorem]{proposition}
\tcolorboxenvironment{proposition}{colback=LightOrange}


\declaretheoremstyle[name=Proof,]{prosty}
\declaretheorem[style=prosty,numberlike=theorem]{proofbox}
\tcolorboxenvironment{proofbox}{colback=LightOrange}

\declaretheoremstyle[name=Axiom,]{prcpsty}
\declaretheorem[style=prcpsty,numberlike=theorem]{axiom}
\tcolorboxenvironment{axiom}{colback=LightGreen}

\declaretheoremstyle[name=Lemma,]{prcpsty}
\declaretheorem[style=prcpsty,numberlike=theorem]{lemma}
\tcolorboxenvironment{lemma}{colback=LightCyan}





\setstretch{1.2}
\geometry{
    textheight=9in,
    textwidth=5.5in,
    top=1in,
    headheight=12pt,
    headsep=25pt,
    footskip=30pt
}

% ------------------------------------------------------------------------------

\begin{document}

% ------------------------------------------------------------------------------
% Cover Page and ToC
% ------------------------------------------------------------------------------

\title{ \normalsize \textsc{}
		\\ [2.0cm]
		\HRule{1.5pt} \\
		\LARGE \textbf{\uppercase{Lecture 7 sep 11}}
		\HRule{2.0pt} \\ [0.6cm] \LARGE{Cosets}
		}

\date{\today}
\author{\textbf{Author} \\ 
		Tom Jeong
        }

\maketitle
\newpage

\tableofcontents
\newpage

% ------------------------------------------------------------------------------
\section{Bijective homomorphism}
A Bijective homomorphism is called an isomorphism. 

\begin{proposition}[2.4.9] \leavevmode \\ 
    Let $f: G \rightarrow K$ be a group homomorphism. \begin{enumerate}
        \item the image $Im f \subseteq K $ is a subgroup of $K$. 
        \item The kernel $Ker f \subseteq G$ is a normal subgroup of $G$.
        \item $f$ is injective if and only if $Ker f = \{e\}$
    \end{enumerate}
\end{proposition}
Proof continutes. \\ 
2. \underline{Normality}
 Let $N = Ker f $ for every $g\ in G$ and $n \in N$, $gng^{-1} \in N$.
 \begin{align}
    f(gng^{-1}) &= f(g)f(n)f(g)^{-1} \\
    &= f(g)f(g)^{-1} \\ 
    &= e \\ 
    &\text{so } \\ 
    gng^{-1} &\in Ker f = N \forall g \in G, n \in N \\ 
    \text{Hence } gNg^{-1} &\subseteq N \text{ }\forall g \in G 
 \end{align}
3. $\rightarrow$ Since $f(e_G) = e_K$ and f is injective. $ker f= \{e_G\}$ \\
$\leftarrow$ Suppose $ker f = \{e_G\}$ and $f(g) = f(h)$ then $f(gh^{-1}) = f(g)f(h)^{-1} = e$ so $gh^{-1} \in ker f = \{e_G\}$ so $g = h$ so $f$ is injective.


\begin{theorem}[2.5.1 Isomorphism Theorem] \leavevmode\\ 
    Leg $G $ and $K$ be groups and $f: G \rightarrow K$ a homomorphism with the kernel $Ker f = N$, then $\tilde{f}: G /N \rightarrow f(G)$  given by $\tilde{f}(gN) = f(g)$ is well defined and a group isomorphism. 
\end{theorem}
\begin{proof}
    Notice that $f(x) = f(y) \iff f(y)^{-1} f(x) = e_K \iff f(y^{-1}x) = e_K \iff y^{-1}x \in N \iff xN = yN$ so $\tilde{f}$ is well defined. \\
   \\ Recall lemma 2.2.6 -ii that $y^{-1}x \in N \iff xN = yN$ so $\tilde{f}$. Hence $$f(x) = f(y) \iff xN = yN$$ \\ 
   $\rightarrow$ gives that $\tilde{f}$ is injective. \\
   $\leftarrow$ give that well defined
   \end{proof}

\begin{proposition}[2.4.9] \leavevmode \\ 
    states that $Ker f = N$ is normal so $\tilde{f}$ is a homomorphism since $\tilde{f}((g_1N)( g_2N)) = \tilde{f}(g_1 g_2 N) = f(g_1 g_2) = f(g_1) f(g_2) = \tilde{f}(g_1 N) \tilde{f}(g_2 N)$ \\ 
    Now $\tilde{f}$ is surjective because $f$ is surjective onto $f(G)$ so $\tilde{f}$ is an isomorphism. 
    
\end{proposition}
 \end{document}
 