\documentclass{article}
\usepackage{amsmath, amsthm, amssymb, amsfonts}
\usepackage{thmtools}
\usepackage{graphicx}
\usepackage{setspace}
\usepackage{geometry}
\usepackage{float}
\usepackage{hyperref}
\usepackage[utf8]{inputenc}
\usepackage[english]{babel}
\usepackage{framed}
\usepackage[dvipsnames]{xcolor}
\usepackage{tcolorbox}
\usepackage{amsmath}
\usepackage{array}
\usepackage{multirow}


\colorlet{LightGray}{White!90!Periwinkle}
\colorlet{LightOrange}{Orange!15}
\colorlet{LightGreen}{Green!15}

\newcommand{\HRule}[1]{\rule{\linewidth}{#1}}

\colorlet{LightGray}{black!10}
\colorlet{LightOrange}{orange!15}
\colorlet{LightGreen}{green!15}
\colorlet{LightBlue}{blue!15}
\colorlet{LightCyan}{cyan!15}



\declaretheoremstyle[name=Theorem,]{thmsty}
\declaretheorem[style=thmsty,numberwithin=section]{theorem}
\usepackage{tcolorbox} % Add missing package
\tcolorboxenvironment{theorem}{colback=LightGray}

\declaretheoremstyle[name=Definition,]{thmsty}
\declaretheorem[style=thmsty,numberwithin=section]{definition}
\tcolorboxenvironment{definition}{colback=LightBlue}

\declaretheoremstyle[name=Proposition,]{prosty}
\declaretheorem[style=prosty,numberlike=theorem]{proposition}
\tcolorboxenvironment{proposition}{colback=LightOrange}


\declaretheoremstyle[name=Proof,]{prosty}
\declaretheorem[style=prosty,numberlike=theorem]{proofbox}
\tcolorboxenvironment{proofbox}{colback=LightOrange}

\declaretheoremstyle[name=Axiom,]{prcpsty}
\declaretheorem[style=prcpsty,numberlike=theorem]{axiom}
\tcolorboxenvironment{axiom}{colback=LightGreen}

\declaretheoremstyle[name=Lemma,]{prcpsty}
\declaretheorem[style=prcpsty,numberlike=theorem]{lemma}
\tcolorboxenvironment{lemma}{colback=LightCyan}





\setstretch{1.2}
\geometry{
    textheight=9in,
    textwidth=5.5in,
    top=1in,
    headheight=12pt,
    headsep=25pt,
    footskip=30pt
}

% ------------------------------------------------------------------------------

\begin{document}

% ------------------------------------------------------------------------------
% Cover Page and ToC
% ------------------------------------------------------------------------------

\title{ \normalsize \textsc{}
		\\ [2.0cm]
		\HRule{1.5pt} \\
		\LARGE \textbf{\uppercase{Lecture 7 sep 11}}
		\HRule{2.0pt} \\ [0.6cm] \LARGE{Cosets}
		}

\date{\today}
\author{\textbf{Author} \\ 
		Tom Jeong
        }

\maketitle
\newpage

\tableofcontents
\newpage

% ------------------------------------------------------------------------------
\section{Bijective homomorphism}
A Bijective homomorphism is called an isomorphism. 

\begin{proposition}[2.4.9] \leavevmode \\ 
    Let $f: G \rightarrow K$ be a group homomorphism. \begin{enumerate}
        \item the image $Im f \subseteq K $ is a subgroup of $K$. 
        \item The kernel $Ker f \subseteq G$ is a normal subgroup of $G$.
        \item $f$ is injective if and only if $Ker f = \{e\}$
    \end{enumerate}
\end{proposition}
Proof continutes. \\ 
2. \underline{Normality}
 Let $N = Ker f $ for every $g\ in G$ and $n \in N$, $gng^{-1} \in N$.
 \begin{align}
    f(gng^{-1}) &= f(g)f(n)f(g)^{-1} \\
    &= f(g)f(g)^{-1} \\ 
    &= e \\ 
    &\text{so } \\ 
    gng^{-1} &\in Ker f = N \forall g \in G, n \in N \\ 
    \text{Hence } gNg^{-1} &\subseteq N \text{ }\forall g \in G 
 \end{align}
3. $\rightarrow$ Since $f(e_G) = e_K$ and f is injective. $ker f= \{e_G\}$ \\
$\leftarrow$ Suppose $ker f = \{e_G\}$ and $f(g) = f(h)$ then $f(gh^{-1}) = f(g)f(h)^{-1} = e$ so $gh^{-1} \in ker f = \{e_G\}$ so $g = h$ so $f$ is injective.


\begin{theorem}[2.5.1 Isomorphism Theorem] \leavevmode\\ 
    Leg $G $ and $K$ be groups and $f: G \rightarrow K$ a homomorphism with the kernel $Ker f = N$, then $\tilde{f}: G /N \rightarrow f(G)$  given by $\tilde{f}(gN) = f(g)$ is well defined and a group isomorphism. 
\end{theorem}
\begin{proof}
    Notice that $f(x) = f(y) \iff f(y)^{-1} f(x) = e_K \iff f(y^{-1}x) = e_K \iff y^{-1}x \in N \iff xN = yN$ so $\tilde{f}$ is well defined. \\
   \\ Recall lemma 2.2.6 -ii that $y^{-1}x \in N \iff xN = yN$ so $\tilde{f}$. Hence $$f(x) = f(y) \iff xN = yN$$ \\ 
   $\rightarrow$ gives that $\tilde{f}$ is injective. \\
   $\leftarrow$ give that well defined
   \end{proof}

\begin{proposition}[2.4.9] \leavevmode \\ 
    states that $Ker f = N$ is normal so $\tilde{f}$ is a homomorphism since $\tilde{f}((g_1N)( g_2N)) = \tilde{f}(g_1 g_2 N) = f(g_1 g_2) = f(g_1) f(g_2) = \tilde{f}(g_1 N) \tilde{f}(g_2 N)$ \\ 
    Now $\tilde{f}$ is surjective because $f$ is surjective onto $f(G)$ so $\tilde{f}$ is an isomorphism. 
    
\end{proposition}
example: $D_3$ is the symmetries of a triangle; define $f: D_3 \rightarrow  \mathbb{Z} / 2 \mathbb{Z}$ by sending rotations (including e) to [0] and [1]
\\ 
$Ker f = \{e, r_1, r_2 \} = N $ so by Isomorphism Theorem, $\tilde{f}: D_3 / N \rightarrow \mathbb{Z} / 2\mathbb{Z}$ is an isomorphism. \\ 
$\{e, r_1, r_2\} = eN \rightarrowtail [0]$ \\
$\{s_1, s_2, s_3\} = s_1N \rightarrowtail [1]$ \\


Examples: det $(GL_2(\mathbb{R}), \cdot) \rightarrow (\mathbb{R}^*, \cdot)$ is a homomorphism. the kernel is the set of matrices with determinant 1. 

$ker det = \{A \in GL_2(\mathbb{R}) | det A = 1\} = SL_2(\mathbb{R})$ is a normal subgroup of $GL_2(\mathbb{R})$ and $SL_2(\mathbb{R}) \cong \mathbb{R}^*$
$$\tilde{det}: GL_2(\mathbb{R} / SL_2(\mathbb{R}) ) \rightarrow \mathbb{R}^*$$ is an isomorphism

Another example. Let 
$K = \{z \in \mathbb{C} | |z| = 1\}$ be the unit circle in $\mathbb{C}$ and $G = \mathbb{R}$ be the group of real numbers under addition.
$(K, \cdot) $ is a group and define $f : (\mathbb{R }, + ) \rightarrow (K, \cdot)$ by $f(x) = e^{2\pi i x}$ By Isom Theorem $$\mathbb{R} / 2\pi\mathbb{Z} \cong K$$

Consider $g \in G, n\in \mathbb{Z}^+$, $g^n = \underbrace{g \cdot g \cdot \ldots \cdot g}_{n \text{ times}}$ and $g^{-n} = \underbrace{g^{-1} \cdot g^{-1} \cdot \ldots \cdot g^{-1}}_{n \text{ times}}$ so $g^n \cdot g^{-n} = e_G$


\begin{proposition}[2.6.1] \leavevmode \\ 
    Let $G$ be a group and $g \in G$ \\ 
    $f_g: \mathbb{Z} \rightarrow G$ \\ 
    $f_g(n) \rightarrowtail g^n$ \\
    is a homomorphism
    
\end{proposition}

\begin{definition}
    \begin{enumerate}
        \item the image of $f_g$ is the cyclic subgroup of $G$ generated by $g$ and is denoted $\langle g \rangle$
        \item we will call |$\langle g \rangle$| the order of $g$ and write $|g|$ or ord G
        \item Alternativley the order of element $g\in G $ is the smallest posirie integer n such that $g^n = e$
    \end{enumerate}
\end{definition}
\underline{Note} If $g^n \not = e$ for only $n \in \mathbb{Z}^+$ then $|g| = \infty$ \\
\begin{definition}
    $D_3$ order of $r_1 = 3$ and order of $s_1 = 2$
\end{definition} 
\begin{proposition}[2.6.3] \leavevmode \\ 
    Let $G$ be a finite group and $g \in G$ . 
    \begin{enumerate}
        \item the order ord $g$ of g divides the order of $G$
        \item $ g^{|G|} = e$
        \item if $g^n = e$ then for some positive n, $ord(g) | n$
    \end{enumerate}
    
    
\end{proposition}
\begin{proof}
    \leavevmode \\ 
    \begin{enumerate}
        \item $ord(g) = | \langle g \rangle |$ and BY Lagrange's Theorem $|G| = | \langle g \rangle | \cdot [G : \langle g \rangle ]$ so $ord(g) | |G|$
        \item $g^{|G|} = g^{| \langle g \rangle | \cdot [G : \langle g \rangle ]} = (g^{ord(g)})^{[G : \langle g \rangle ]} = e^{[G : \langle g \rangle ]} = e$
        \item If $g^n  = e$ then $n \in ker f_g = n_g \mathbb{Z}$ for some $n_g \not = 0 $ so we have that $n_g = ord(g)$ so $ord(g) | n$a
    \end{enumerate}
\end{proof}
 \end{document}
 