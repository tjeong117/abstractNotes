\documentclass{article}
\usepackage{amsmath, amsthm, amssymb, amsfonts}
\usepackage{thmtools}
\usepackage{graphicx}
\usepackage{setspace}
\usepackage{geometry}
\usepackage{float}
\usepackage{hyperref}
\usepackage[utf8]{inputenc}
\usepackage[english]{babel}
\usepackage{framed}
\usepackage[dvipsnames]{xcolor}
\usepackage{tcolorbox}
\usepackage{amsmath}
\usepackage{array}
\usepackage{multirow}


\colorlet{LightGray}{White!90!Periwinkle}
\colorlet{LightOrange}{Orange!15}
\colorlet{LightGreen}{Green!15}

\newcommand{\HRule}[1]{\rule{\linewidth}{#1}}

\colorlet{LightGray}{black!10}
\colorlet{LightOrange}{orange!15}
\colorlet{LightGreen}{green!15}
\colorlet{LightBlue}{blue!15}
\colorlet{LightCyan}{cyan!15}



\declaretheoremstyle[name=Theorem,]{thmsty}
\declaretheorem[style=thmsty,numberwithin=section]{theorem}
\usepackage{tcolorbox} % Add missing package
\tcolorboxenvironment{theorem}{colback=LightGray}

\declaretheoremstyle[name=Definition,]{thmsty}
\declaretheorem[style=thmsty,numberwithin=section]{definition}
\tcolorboxenvironment{definition}{colback=LightBlue}

\declaretheoremstyle[name=Proposition,]{prosty}
\declaretheorem[style=prosty,numberlike=theorem]{proposition}
\tcolorboxenvironment{proposition}{colback=LightOrange}


\declaretheoremstyle[name=Proof,]{prosty}
\declaretheorem[style=prosty,numberlike=theorem]{proofbox}
\tcolorboxenvironment{proofbox}{colback=LightOrange}

\declaretheoremstyle[name=Axiom,]{prcpsty}
\declaretheorem[style=prcpsty,numberlike=theorem]{axiom}
\tcolorboxenvironment{axiom}{colback=LightGreen}

\declaretheoremstyle[name=Lemma,]{prcpsty}
\declaretheorem[style=prcpsty,numberlike=theorem]{lemma}
\tcolorboxenvironment{lemma}{colback=LightCyan}





\setstretch{1.2}
\geometry{
    textheight=9in,
    textwidth=5.5in,
    top=1in,
    headheight=12pt,
    headsep=25pt,
    footskip=30pt
}

% ------------------------------------------------------------------------------

\begin{document}

% ------------------------------------------------------------------------------
% Cover Page and ToC
% ------------------------------------------------------------------------------

\title{ \normalsize \textsc{}
		\\ [2.0cm]
		\HRule{1.5pt} \\
		\LARGE \textbf{\uppercase{Lecture 8 sep 16}}
		\HRule{2.0pt} \\ [0.6cm] \LARGE{Cosets}
		}

\date{\today}
\author{\textbf{Author} \\ 
		Tom Jeong
        }

\maketitle
\newpage

\tableofcontents
\newpage

% ------------------------------------------------------------------------------
\section{Order of Element}
ex. What are the possible orders of an element $[n] \in \mathbb{Z} / 5\mathbb{Z}$ 
$5 = | \mathbb{Z} / 5\mathbb{Z} |$ \\is prime so by Proposition 2.6.3, every element is either order 1 or order 5. \\ \underline{note:} Any order 5 element generates $\mathbb{Z} / 5\mathbb{Z}$

example: what are the possible homomorphism from $\mathbb{Z} / 3\mathbb{Z}$ to $ \mathbb{Z} / 6\mathbb{Z}$? \\
Recall if $f: G \to K $ is a homormorphism then $ker f \leq G$ and $im f \leq K$ \\
\underline{note:} if ord(g) = n then $e_k = f(e_G) = f(g^n) = f(g) ^ n$  \\ so $f(g)$ is of order divides n. by proposition 2.6.3 (3) -- check notes from previous lecture
 \\ 
\begin{align*}
     \mathbb{Z} &/ 3\mathbb{Z} \\ 
     ord([0])& = 1\\
    ord([1]) &= ord([2]) = 3 \\ 
    \mathbb{Z} &/ 6\mathbb{Z} \\
    ord([0]) &= 1 \\
    ord([1]) &= ord([5]) = 6 \\
    ord([2]) &= ord([4]) = 3 \\ 
    ord([3]) &= 2
\end{align*}
 What homomorphisms are possible ?? \\ 
\begin{align*}
    f([0]) &= [0] \\
    f([1]) &= [0] \text{ or }  [2] \text{ or }[4]\\
    f([2]) &= f([1] + [1]) = f([1]) + f([1])
\end{align*}
so there is 3. homo morphisms 
\begin{definition}[2.7.1] \leavevmode \\ 
 A cyclic group is a group $G$ containnig an element $g$ such that $G = <g>$ The element $g$ is called the generator of $G$ and we say that $G$ is the generated by $g$ 
    
\end{definition}
example: $\mathbb{Z} / n\mathbb{Z}$ is generated by $[1]$ or $[n-1]$


Can there be an infinite cycle group? yes; the $\mathbb{Z}$ (under addition) is cyclic that is generated by 1 or -1. 
 \begin{proposition}
    Any cyclic group is isomorphic to $\mathbb{Z} / n\mathbb{Z}$ for some $n \in \mathbb{N}$ \\ \underline{Note:} $0\mathbb{Z} =\{0\}$, $\mathbb{Z} / 0\mathbb{Z} = \mathbb{Z }$, $\mathbb{Z} / 1\mathbb{Z} = \{0\}$

 \end{proposition}
 \begin{proof}  \leavevmode \\ 
    Consider $G = <g>$ thus \begin{align*}
        f_g &: \mathbb{Z} \to G \\ 
        f_g&(n) = g^n 
    \end{align*}
    is a surjective homomorphism then $Ker (f_g) $ is a subgroup of $\mathbb{Z}$ and in  Proiposition 2.2.3, we proved every subgroup of $\mathbb{Z}$ is of the form of $n\mathbb{Z}$ for some $n\in \mathbb{N}$ \\ 
    so $Ker(f_g) = n_g \mathbb{Z}$ for some $n_g \in \mathbb{N}$ and by the first isomorphism theorem, $\mathbb{Z} / n_g \mathbb{Z} \cong G \cong \mathbb{Z} / Ker(f_g) $




\end{proof}

recall from HW the $n^{th}$ roots of unity are: $$z_k = e^{2\pi i k / n} \text{ for } k = 0, 1, 2, \ldots, n-1$$

 \subsection{exercise} 
 the nth roots of unity are a cyclic group generated by $z_1 = e^{2\pi i / n}$ and is isomorphic to $Z / nZ$ 
\\ 
 concrete example and its proof; taking a look at $4$-th roots of unity. draw it: 
    
 $$\{1,i,-1, -i\} \cong (Z / 4Z, + )$$ \begin{align*}
    1 &\to [0] \text{ :  identity}\\
    i &\to [1] \\
    -1 &\to i^2 = [1] + [1] = [2]  \\
    -i &\to [3]
 \end{align*}
    \\ 




\begin{proposition}[2.7.2] \leavevmode \\ 
    A gfroup $G$ of prime order $|G| = p$ is cyclic and isomorphic to $\mathbb{Z} / p\mathbb{Z}$
\end{proposition}
\begin{proof} \leavevmode \\ 
    Let $g \in G$ such that $g \not = e$ and let $H = <g>$. Since $H \leq G$ and by LaGrange Theorem, $|H|$ divides $|G| = p$ so $|H| = 1$ or $p$ but $g \not = e$ so $|H| = p$ and $H = G$ so $G = <g>$ and $G$ is cyclic. \\
\end{proof}

\section*{Cyclic groups of composite order}
Q: What about cyclic groups with orders not prime? \\ 
Lets look at $Z / 12Z$ \\
\begin{align*}
    ord[0] &= 1 \\
    ord[1] &= ord[5] = 12 \\
    ord[2] &= ord[10] = 6 \\
    ord[3] &= ord[9] = 4 \\
    ord[4] &= ord[8] = 3 \\
    ord[6] &= ord[6] = 2 \\ 
    ord[7] &= ord[11] = 12
\end{align*}
we can see a pattern here.
$[1], [5], [7], [11]$ are the generators of $Z / 12Z$ and $Z / 12Z$ is isomorphic to $\mathbb{Z} / 12\mathbb{Z}$

\begin{definition}
    Let $n \in \mathbb{Z}$ the Euler $\phi$ function is defined as $\phi(n) = | \{ k \in \mathbb{Z} | 1 \leq k \leq n \text{ and } gcd(k,n) = 1 \} |$
\end{definition}

\begin{lemma}[Cor 1.5.10] \leavevmode \\
     Let $a, b, c \in \mathbb{Z}$ If gcd (a,b) = 1 and $a | bc$ then $a|c$ 
\end{lemma} see textbook for proof 




\begin{proposition}[2.7.4] \leavevmode \\ 
    let $G$ be a cyclic group 
    \begin{enumerate}
        \item every subgroup of $G$ is cyclic 
        \item Suppose $G$ is finite and is a divisor of $|G|$ Then $G$ contains a unique subgroup $H$ of order $d$ for each $d | |G|$ and this subgroup is cyclic.
    \end{enumerate}
    
\end{proposition}


\end{document}
 